\def\tutdate{09.12.2019}

\documentclass[]{beamer}
\usepackage{../templates/beamerthemekit}
\input{../assets/dependencies}
\input{../assets/definitions}
\title[Grundbegriffe der Informatik]{GBI\\Tutorium 8}
\date{\tutdate}
\subtitle{\tutTitle}
\author{Dennis Kobert dennis@kobert.dev}

\institute{}

\titleimage{bg}
%\titleimage{bg-advent}

%\include{assets/multislider}

\setbeamercovered{invisible}

%\usepackage[citestyle=authoryear,bibstyle=numeric,hyperref,backend=biber]{biblatex}
%\addbibresource{templates/example.bib}
%\bibhang1em

	
\def\tutTitle{Praedikatenlogik}
\begin{document}

\selectlanguage{ngerman}

%title page
\begin{frame}
	\titlepage
\end{frame}

\section{Hinweise}

\begin{frame} {MIMA-Simulator}
https://github.com/weisJ/Mima
\end{frame}

\section{Praedikatenlogik}
\begin{frame}{Grundlagen zu Praedikatenlogik}
	Praedikatenlogik (PL) \ip \markBlue{erweitert} Aussagenlogik durch Ergaenzen von ``Praedikaten'', einer Art von Funktionen, die Wahrheitswerte zurückgeben.

	Alphabet der Praedikatenlogik:
	
	\begin{itemize}
		\pitem $\lnot, \land, \lor, \rightarrow, \leftrightarrow, (, )$, also Alphabet der Aussagenlogik.
		\pitem $\forall$ Allquantor \ip ($\forall x$ heißt ``für alle $x$ gilt...'')
		\pitem $\exists$ Existenzquantor \ip ($\exists x$ heißt ``es existiert min. ein $x$... für das gilt...'')
		\pitem $x,y,z,x_i \in Var_{PL}$ Variablen
		\pitem $c, d, c_i \in Const_{PL}$ Konstanten
		\pitem $f, g, h, f_i \in Fun_{PL}$ Funktionen 
		\pitem $R, S, R_i \in Rel_{PL}$ Relationen (funktionieren aehnlich wie Funktionen)
		\pitem $\objequiv$ Objektgleichheit
		\pitem $,$ Komma
	\end{itemize}
\end{frame}

\begin{frame}{Gliederung der Praedikatenlogik}
	\begin{block}{Terme}
		Ein Term ist ein Element aus der Sprache über $A_{Ter} := \{\markBlue{(}, \markBlue{)}, \markBlue{,}\} \cup Var_{PL} \cup Const_{PL} \cup Fun_{PL}$.
	\end{block}
	
	\begin{block}{Atomare Formeln}
		Atomare Formeln sind zum Beispiel
		\begin{itemize}
			\pitem Objektgleichheiten $f_1 \objequiv f_2$
			\pitem Relation von Termen $R(t_1, t_2, ...)$
		\end{itemize}
	\end{block}

	\begin{block}{Stelligkeit einer Funktion}
		Die Stelligkeit $ar(f) \in \N_+$ einer Funktion gibt die Anzahl der Parameter von $f$ an. \ip (Analog Stelligkeit von Relationen $ar(R)$)
	\end{block}
	
	%\bp
\end{frame}

\begin{frame}{Verstaendnis von Termen, Atomaren Formeln, Stelligkeit}
\begin{itemize}
	\item Woraus kann ein Term bestehen? \pause
	\item[$\rightarrow$] Aus Klammern $\markBlue{(}, \markBlue{)}$, Kommas $\markBlue{,}$, Variablen, Konstanten, Funktionen.\pause
	\item Was davon sind atomare Formeln: $R(x) \land S(f(x, c))$, $R(x, g(c, f(y, x))$?\pause
	\item[$\rightarrow$] Nein, ja.\pause
	\item Was sind die Stelligkeiten folgender Funktionen: $f(a, b, c), g(a), h(a, b)$? \pause \item[$\rightarrow$] $3,1,2$.
\end{itemize}	
\end{frame}

\begin{frame}{Grammatik der Praedikatenlogik}
	Praedikatenlogische Formeln werden durch die Grammatik $G := (N_{Ter}, A_{Ter}, T, P_{Ter})$ erzeugt mit:
	
	\bp
	
	\begin{itemize}
		\pitem $m+1$ Nichtterminalsymbolen $N_{Ter} := \{T\} \cup \{L_i | i \in \N_+ \text{ und } i \leq m \}$ ($m = $ Maximale Stelligkeit von Funktionen)
		\pitem Terminalsymbolen: Alphabet, aus dem Terme erzeugbar sind
		%\pitem Startsymbol $T$
		\pitem Produktionen
		\begin{alignat*}{2}
		L_{i+1} &\to L_i , T &\qquad& \text{für jedes } i\in\N_+\text{ mit } i<m   \\
		L_1  &\to T \\ % ACHTUNG Komma ist hier ein Terminalsymbol, kein Trennsymbol
		T &\to c_i && \text{für jedes } c_i\in Const_{PL}\\
		T &\to x_i && \text{für jedes } x_i\in Var_{PL}\\
		T &\to f_i(L_{ar(f_i)} ) && \text{für jedes } f_i\in Fun_{PL}
		\end{alignat*}
	\end{itemize}

	\bp
	
	Beispiel: Seien Funktionen $f,g$ mit $ar(f) = 2, ar(g) = 1$, Konstante $c$ und Variablen $x,y$ gegeben. Was kann man damit machen?
\end{frame}

\begin{frame}{Grammatik der Praedikatenlogik}
	
	
	\begin{columns}
		\begin{column}{0.6\textwidth}
			Beispiel: Seien Funktionen $f,g$ mit $ar(f) = 2, ar(g) = 1$, Konstante $c$ und Variablen $x,y$ gegeben. Was kann man damit machen?\vspace{.2cm}
			
			\bp
			
			Dann: 
			\begin{itemize}
				\item $N_{Ter}=\{ T, L_1, L_2 \}$
				\item $\begin{aligned}[t]
				P_{Ter} = \{  L_2 & \to L_1 , T \\% ACHTUNG Komma ist hier ein Terminalsymbol, kein Trennsymbol
				L_1 & \to T  \\
				T   & \to c \\
				T   & \to x \\
				T   & \to y \\
				T   & \to g ( L_1 ) \\
				T   & \to f ( L_2 ) \}
				\end{aligned}
				$
			\end{itemize}
		\end{column}
		
		\begin{column}{0.4\textwidth}
			\bp
			
			\begin{taskblock}{Aufgabe zu Grammatiken und Praedikatenlogik}
				Welche dieser Formeln entsprechen dieser Grammatik?
				\vspace{.2cm}
				\begin{itemize}
					\pitem $f(c, g(x))$
					\pitem $f(x, y, c)$
					\pitem $g(f(c, c))$
					\pitem $g(g(f(g(x), g(f(c, c))))$
					\pitem $g(c, f)c)$
				\end{itemize}
			
				\ip Bilde die Ableitungsbaeume zu den korrekten Formeln.
			\end{taskblock}
		\end{column}
	\end{columns}
\end{frame}

\begin{frame}{Bindungsstaerken}
	\begin{block}{Bindungsstaerke}
		Verschiedene Operanden ``binden'' staerker als andere. \ip Wenn ein Operand staerker als die umliegenden Operanden bindet, tritt derselbe Effekt auf, wie wenn Klammerung geschehen würde.
	\end{block}

	\bp
	
	Bindungsstaerken absteigend:
	\begin{itemize}
		\ip \item $\forall / \exists\ip,\lnot\ip,\land\ip,\lor\ip,\rightarrow/\leftarrow\ip,\leftrightarrow$
	\end{itemize}

	\bp
	
	Finde aequivalente Formeln, die mit möglichst wenig Klammern auskommen:
	\begin{itemize}
        \pitem $\exists x \forall y (R(f(x), g(x))) \lor \forall z R(c, x)$ %TODO finish
	\end{itemize}

\end{frame}
% TODO ich war hier
\begin{frame}{Quantoren}
\begin{itemize}
	\item $\forall x p(x)$ heißt\ip: für alle $x \in D$ gilt die Aussage $p(x)$.
	\pitem $\exists x p(x)$ heißt\ip: für (mindestens) ein $x \in D$ gilt die Aussage $p(x)$.
	\pitem Gilt $\forall x \exists y \quad p(x,y) = \exists y \forall x \quad p(x,y)$?
	\begin{itemize}
		\pause\item Zum Beispiel $p(x,y) := $``Person $x$ ist mit Person $y$ verheiratet.''
		\pause\item Also:
		\begin{itemize}
			\pitem $\forall x \exists y \quad p(x,y) = $ Für jede Person $x$ gibt es eine Person $y$, mit der $x$ verheiratet ist.
			\pitem $\exists y \forall x \quad p(x,y) = $ Es gibt eine Person $y$, sodass für alle Personen $x$ gilt, dass $x$ mit $y$ verheiratet ist.
		\end{itemize}
		\pitem Eher nicht. Reihenfolge ist also wichtig!
	\end{itemize}
\end{itemize}
\end{frame}

\begin{frame}{Bindungsbereich von Quantoren}
	Quantoren binden Variablen nur zu der zugehörigen Teilformel.
	
	\bp
	
	\begin{itemize}
		\item Zum Beispiel: $p(x) \land \markBlue{\forall x \exists y} \markGreen{(p(x) \land q(x,y,z)} \rightarrow r(x) $
		\pitem Welcher Teil der Formel muss für alle $x$ gelten? Welcher für $y$?
		\pitem Variablen, die nicht im Wirkungsbereich eines Quantors liegen, nennt man \markGreen{frei}.
	\end{itemize}

	\bp Überschattung ist möglich, durch Quantoren definierte Variablen beziehen sich immer auf den \markBlue{naechsten} Quantor.
	\begin{itemize}
		\pitem Ist $\forall x (p(x) \land \forall x (\lnot p(x))))$ erfüllbar?
		\pause\item Ja: $\forall x (p(x) \land \forall \hat{x} (\lnot p(\hat{x}))))$ 
	\end{itemize}
\end{frame}

\begin{frame}{Bindungsbereich von Quantoren}
	Substitution ist möglich. Dabei wird eine \markBlue{freie} Variable durch einen Term ersetzt, die Substitution wird mit $\sigma[a/b]$ bezeichnet, wobei $a$ durch $b$ ersetzt wird.
	
	\vertspace
	
	%TODO alternative Schreibweise für Substitution: \sigma_{x/y}
	
	\bp
	
	Führe die folgenden Substitutionen durch:
	\begin{itemize}
		\item<1-> \(\sigma[x/5] (p(x) \lor q(x,y))\)
		\item<4-> \(= p(5) \lor q(5,y)\)
		\item<2-> \(\sigma[x/5] (p(x) \lor \forall x (q(x,y))\)
		\item<5-> \(= p(5) \lor \forall x (q(x,y))\)
		\item<3-> \(\sigma[x/y, y/x, z/f(z)] (p(z) \land q(x,y))\)
		\item<6-> \(= p(f(z)) \land q(y,x)\)
	\end{itemize}

	\vertspace
	
	\bp
	
	Welche der Variablen sind gebunden (und durch welche Quantoren), welche sind frei? 
	\begin{itemize}
		\pitem $p(x) \rightarrow \forall x \exists y (p(x) \land q(y,z) \leftrightarrow \forall z (q(x,z)))$
		\pitem $\forall y(p(f(x,y))) \lor \exists z(q(z,f(y,z)))$
	\end{itemize}
\end{frame}

\begin{frame}{Kollision}
Eine Kollision liegt vor, wenn eine freie Variable durch eine Substitution gebunden wird. \pause \newline
Beispiel: $\sigma[z/x] (\forall x (p(x, z)))$ \pause \newline
Sind folgende Substitutionen kollisionsfrei?
\begin{itemize}
	\item $\sigma[x/y, y/f(x,z)](\forall x(p(x) \rightarrow q(x,y)) \rightarrow p(x))$
	\item $\sigma[x/g(z), y/x, z/y](\forall z(p(z) \land \forall x(q(x,y) \land \exists y (p(y)))))$
	\item $\sigma[y/z, z/f(z)](q(y,z) \leftrightarrow \exists y(p(y)))$
	\pitem Nur 3 ist kollisionsfrei
\end{itemize}
\end{frame}


\begin{frame}{Semantik von praedikatenlogischen Formeln}
	Um praedikatenlogische Formeln zu interpretieren, brauchen wir einige neue Mengen:
	
	\begin{itemize}
		\pitem Interpretation $(D, I)$\ip, bestehend aus...
		\begin{itemize}
			\pitem Universum $D \neq \emptyset$ mit...
			\begin{itemize}
				\pitem $I(c_i) \in D$ für $c_i \in Const_{PL}$
				\pitem $I(f_i) : D^{ar(f_i)} \rightarrow D$ für $f_i \in Fun_{PL}$
				\pitem $I(R_i) \subseteq D^{ar(R_i)}$ für $R_i \in Rel_{PL}$
				\pitem $I$ weißt also den Komponenten Bedeutungen zu, ``definiert diese''
			\end{itemize}
			
			\pitem Variablenbelegung $\sigma : Var_{PL} \rightarrow D$, z.B. $\sigma(x) := 3, \sigma(y) := 11$
			\begin{itemize}
				\pitem $\sigma$ definiert also Variablenwerte
			\end{itemize}
		\end{itemize}
		
		\bp
		
		\item Damit können wir praedikatenlogische Formeln definieren!
	\end{itemize}
	
	\p
	
	\begin{block}{$\pval$}
		Die Funktion $\pval : L_{Ter} \cup L_{For} \rightarrow D \cup \B$ weißt einer praedikatenlogischen Formel eine Bedeutung 
		(Wahrheitsgehalt für Formeln und Element des Universums für Terme) zu.
	\end{block}
\end{frame}


\begin{frame}{Beispiel zur Semantik}
	Unterschied zwischen $\pval$ und $I$? \pause $I$ weist Einzelteilen (Konstanten, Funktionen, Relationen) eine Bedeutung zu und $\pval$ einer ganzen Formel.\pause\vertspace
	
	Beispiel:\vertspace\ip
	
	Sei $D := \N_0, I(c) := 10, ar(f) := 2, ar(p) := 1, ar(q) := 2, \sigma(x) := 7$.\ip
	
	Sei $I(f) : \N_0^2 \rightarrow \N_0, (x,y) \mapsto x-y$.\ip
	
	Sei $ar(R) := 2, I(R) := \{(x,y) | x \leq y\}$.\ip
	
	Sei $I(p(x)) = w :\Leftrightarrow x \geq 5, I(q(x,y)) = w :\Leftrightarrow x \geq y$.
\end{frame}

\begin{frame}{Beispiel zur Semantik}
	
	Sei $D := \N_0, I(c) := 10, ar(f) := 2, ar(p) := 1, ar(q) := 2, \sigma(x) := 7$.
	
	Sei $I(f) : \N_0^2 \rightarrow \N_0, (x,y) \mapsto x-y$.
	
	Sei $ar(R) := 2, I(R) := \{(x,y) | x \leq y\}$.
	
	Sei $I(p(x)) = w :\Leftrightarrow x \geq 5, I(q(x,y)) = w :\Leftrightarrow x \geq y$.
	
	\begin{itemize}
		\item $T_1 := p(x) \rightarrow \exists y (q(y,x) \land p(y))$, was ist $\pval(T_1)$?
		
		\pause
		\begin{itemize}
			\item Waehle $y = 8 \in \N_0$. \ip Dann: $I(q(8,7)) = w\ip, I(p(8)) = w$\ip, also $\pval(\exists y (q(y,x) \land p(y))) = w$\ip, und $\pval(T_1) = w$.
		\end{itemize}
		
		\pause
		
		\item $T_2 := p(x) \rightarrow \exists y (q(f(c,y), x) \land p(y))$, was ist $\pval(T_2)$?
		
		\pause
		\begin{itemize}
			\item $\pval(p(x)) = w$
			\ip\item $\pval(q(f(c,y), x)) \ip = \pval(q(f(10,y), x))\ip = \pval(q(10-y, 7))\ip = w$ für $y \in \{0,1,2,3\}$.
			\ip\item $\pval(p(y)) = w$ für $y \geq 5$.
			\ip\item Also: $\pval(T_2) = f$.
		\end{itemize}
	\end{itemize}
\end{frame}

\begin{frame}{Aufgaben zu Praedikatenlogik}
	\begin{taskblock}{Aufgaben zu Praedikatenlogik}
		Gegeben sind folgende Praedikate:
		\begin{itemize}
			\item $Vater(x,y) := $ wahr, gdw. $x$ Vater von $y$ ist, analog $Mutter(x,y)$.
			\item $Maennlich(x) := $ wahr, gdw. $x$ maennlich ist, analog $Weiblich(x)$.
			\item $Verheiratet(x,y) := $ wahr, gdw. $x$ und $y$ verheiratet sind.
		\end{itemize}
	
	
		Drücke die folgenden Aussagen mit praedikatenlogischen Formeln aus:
		
		\begin{itemize}
			\pitem Jede maennliche Person hat eine Mutter.
			\begin{itemize}
				\pause\item $\forall x \exists y (Maennlich(x) \rightarrow Mutter(y,x))$
				\pause\item Kann eine Person jetzt auch mehr als eine Mutter haben? \pause Widerspricht das der ursprünglichen Aussage?
			\end{itemize}
			\pitem Jeder Mann hat Kinder (plural!).
			\begin{itemize}
				\pause\item $\forall x \exists y \exists z (Maennlich(x) \rightarrow ( Vater(x,y) \land Vater(x,z) \land \lnot (y \objequiv z)))$
			\end{itemize}
		\end{itemize}
	\end{taskblock}
\end{frame}

\begin{frame}{Aufgaben zu Praedikatenlogik}
	\begin{taskblock}{Aufgaben zu Praedikatenlogik}
		Gegeben sind folgende Praedikate:
		\begin{itemize}
			\item $Vater(x,y) := $ wahr, gdw. $x$ Vater von $y$ ist, analog $Mutter(x,y)$.
			\item $Maennlich(x) := $ wahr, gdw. $x$ maennlich ist, analog $Weiblich(x)$.
			\item $Verheiratet(x,y) := $ wahr, gdw. $x$ und $y$ verheiratet sind.
		\end{itemize}
		
		
		Drücke die folgenden Aussagen mit praedikatenlogischen Formeln aus:
		
		\begin{itemize}
			\pitem Jede Frau ist mit höchstens einem Mann verheiratet.
			\begin{itemize}
				\pause\item $\forall x \forall y \forall z (Weiblich(x) \land((Maennlich(y) \land Maennlich(z) \land \lnot (y \objequiv z) \land Verheiratet(x,y)) \rightarrow \lnot Verheiratet(x,z)))$
			\end{itemize}
			\pitem Wer maennlich ist, ist nicht weiblich und umgekehrt.
			\begin{itemize}
				\pause\item $\forall x (Maennlich(x) \rightarrow \lnot Weiblich(x) \land Weiblich(x) \rightarrow \lnot Maennlich(x))$
			\end{itemize}
		\end{itemize}
	\end{taskblock}
\end{frame}


\begin{frame}
	\includegraphics[width=\linewidth]{../images/thatsall.png}
\end{frame}


\end{document}
