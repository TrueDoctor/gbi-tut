\def\tutdate{06.11.2019}

%\documentclass[handout]{beamer}
\documentclass{beamer}
\usepackage{../templates/beamerthemekit}
%\usepackage{enumitem}
\input{../assets/dependencies}
\input{../assets/definitions}
\title[Grundbegriffe der Informatik]{GBI\\Tutorium 8}
\date{\tutdate}
\subtitle{\tutTitle}
\author{Dennis Kobert dennis@kobert.dev}

\institute{}

\titleimage{bg}
%\titleimage{bg-advent}

%\include{assets/multislider}

\setbeamercovered{invisible}

%\usepackage[citestyle=authoryear,bibstyle=numeric,hyperref,backend=biber]{biblatex}
%\addbibresource{templates/example.bib}
%\bibhang1em

	
\def\tutTitle{Vollständige Induktion, Formale Sprachen, Übersetzungen/Kodierungen}
\begin{document}

\selectlanguage{ngerman}

%title page
\begin{frame}
	\titlepage
\end{frame}

\section{Hinweise}

\begin{frame} {Häufige Fehler}
	\begin{itemize}
		\item Set comprehension: $B = \{x \in A | P(x) \}$ mit A eine Menge und P eine Aussage
		\item Beispiel: $M = \{n \in \N | n$ $gerade \}$
		\item Sei A ein Alphabet; $x, y \in A; v \in A*$
		\item Warum ist $f(xvy) = xf(v)$ wohldefiniert, auch für f(xy)?
	\end{itemize}
\end{frame}

\section{Vollständige Induktion}
\begin{frame} {Wahrheitsgehalt von unendlich Aussagen}
	Beispielsituation: \p Wir haben unendlich viele Dominosteine. \p Behauptung: \p Alle Dominosteine fallen um.
	
	\begin{itemize}
		\pitem Wir haben Aussagen: \{``1. Stein fällt um'', ``2. Stein fällt um'', ...\}
		\pitem Wie zeigen wir unendlich viele Aussagen?
		\pitem Stelle Aussagen in Abhängigkeit einer Laufvariable $n$ dar:
		\begin{itemize}
			\pitem $A(n) := $ ``$n$-ter Stein fällt um'' $\forall n \in \N$.
		\end{itemize}
		\pitem Aussage $A := $ ``Alle Steine fallen um'' $\equiv$ $A(i)$ ist wahr $\forall i \in \N$.
	\end{itemize}

	\p Wir haben immernoch unendlich viele Aussagen...
	
	\begin{itemize}
		\pitem Zeige: $A(1)$ ist wahr\p , sowie $A(i)\text{ gilt } \rightarrow A(i+1)\text{ gilt }$ für beliebiges $i \in \N$.
		\pitem Also: \p Der erste Stein fällt, sowie: \p falls der $i$-te Stein fällt, so fällt auch der $i+1$-te Stein.
		\pitem Nach dem Prinzip der vollständigen Induktion fallen dann alle Steine um.
	\end{itemize}
\end{frame}

\begin{frame}{Vollständige Induktion}
	\begin{itemize}
		\pitem Beweisverfahren
		\pitem In der Regel zu zeigen: Eine Aussage gilt für alle $n \in \mathbb{N}_+$, manchmal auch für alle $n \in \mathbb{N}_0$
		\pitem Man schließt ``induktiv'' von einem n auf n+1
		\pitem Idee: Wenn die Behauptung für ein beliebiges \textbf{festes} n gilt, dann gilt sie auch für den Nachfolger n+1 (und somit auch für dessen Nachfolger und schließlich für alle n)
	\end{itemize}
\end{frame}

\begin{frame}{Struktur des Beweises}
	\pause 
	Behauptung: (\textit {kurz} \textbf{Beh.:})\\
	Beweis: (\textit{kurz} \textbf{Bew.:})
	\begin{itemize}
		\pause
		\item Induktionsanfang: (\textit{kurz} \textbf{IA:})
		\begin{itemize}
			\item Zeigen, dass Behauptung für Anfangswert gilt (oft $n=1$)
			\item Auch mehrere (z.B. zwei) Anfangswerte möglich
		\end{itemize}
		\pause
		\item Induktionsvoraussetzung: (\textit{kurz} \textbf{IV:})
		\begin{itemize}
			\item Sei $n \in \mathbb{N}_+$ (bzw. $n \in \mathbb{N}_0$) fest aber beliebig und es gelte [Behauptung einsetzen]
		\end{itemize}
		\pause
		\item Induktionsschritt: (\textit{kurz} \textbf{IS:})
		\begin{itemize}
			\item Behauptung für n+1 auf n zurückführen
			\item Wenn induktive Definition gegeben: verwenden!
			\item Sonst: Versuche Ausdruck, in dem (n+1) vorkommt umzuformen in einen Ausdruck, in dem nur n vorkommt
		\end{itemize}
	\end{itemize}

	\p Vorhin:  \p
	\begin{align*}
	\underbrace{A(1) \text{ ist wahr }}_{IA} \text{, sowie } \underbrace{A(i)\text{ gilt }}_{IV} \rightarrow \underbrace{A(i+1) \text{ gilt }}_{IS} \text{ für beliebiges i }\in \N
	\end{align*}
\end{frame}
%Zwei Beispiele, Lösungen siehe gleicher Ordner
% https://www.cl.cam.ac.uk/~mgk25/kuhn-fa.pdf
%Einfacheres S.8
\begin{frame}{Übung zu Vollständiger Induktion}
	\textbf{Aufgabe}\\
	\begin{eqnarray*}
		&x_0 := 0\\
		&\text{Für alle } n \in \mathbb{N}_0: x_{n+1} := x_n + 2n +1
	\end{eqnarray*}			
	\textit{Zeige mithilfe vollständiger Induktion, dass für alle} $n \in \mathbb{N}_0$ \\
	\begin{center}$x_n = n^2$\end{center}
	gilt.
\end{frame}

\begin{frame}{Übung zu vollständiger Induktion}
	\begin{taskblock}{Übungsaufgaben}
		Zeige die Wahrheit folgender Aussagen mit vollständiger Induktion:
		
		\begin{itemize}
			\item $\sum\limits_{i=1}^n i = \frac{n(n+1)}{2}$ $\forall n \in \N$.
			\item $\sum\limits_{i=1}^n i^2 = \frac{n(n+1)(2n+1)}{6}$ $\forall n \in \N$
		\end{itemize}
	\end{taskblock}
\end{frame}

\section{Formale Sprache}

\begin{frame}{Formale Sprache}
	\begin{itemize}
		\pitem Was war nochmal $A^*$? Menge aller Wörter \emph{beliebiger} Länge über Alphabet $A$.
		\pitem Was war nochmal eine formale Sprache?
	\end{itemize}
	
	\pause
	
	\begin{block}{Formale Sprache}
		Eine Formale Sprache $L$ über einem Alphabet $A$ ist eine Teilmenge $L \subseteq A^*$.
	\end{block}

	\pause Als Beispiel von vorigen Folien:
	
	\begin{itemize}
		\pitem $A := \{b, n, a\}$.
		\begin{itemize}
			\pitem $L_1 := \{ban, baan, nba, aa\}$ ist eine mögliche formale Sprache über $A$.
			\pitem $L_2 := \{banana, bananana, banananana, ...\}$ \pause $ = \{w : w = bana(na)^k, k \in \N\}$ auch.
			\pitem $L_3 := \{ban, baan, baaan, ...\}$ auch. \pause Andere Schreibweise? \pause \\ $ L_3 = \{w : w = ba^kn, k \in \N \}$
		\end{itemize}
	\end{itemize}
\end{frame}

\begin{frame}{Produkt von Sprachen}
	\begin{block}{Produkt von formalen Sprachen}
		Von zwei formalen Sprachen $L_1, L_2$ lässt sich das Produkt $L_1 \cdot L_2$ bilden mit \pause $L_1 \cdot L_2 = \{w_1w_2 : w_1 \in L_1 $ und $w_2 \in L_2 \}$.
	\end{block}

	Sei $A := \{a, b\}, B := \{\alpha, \beta, \gamma, \epsilon, \delta\}$.
	
	\begin{itemize}
		\pitem Sprache $L_1 \subseteq A^*$, die zuerst drei $a$'s enthält und dann entweder zwei $b$'s oder vier $a$'s? \pause $L_1 = \{aaa\}\cdot\{bb,aaaa\}$.
		\pitem Sprache $L_2 \subseteq A^*$, die alle Wörter über $A$ enthält außer $\epsilon$? \pause $L_2 = A \cdot A^* \pause = A^* \backslash \{\epsilon\}$.
		\pitem Sprache $L_3 \subseteq B^*$, die alle Wörter über $B$ enthält, mit:
		\begin{itemize}
			\p\item Zwei beliebigen Zweichen aus B.
			\p\item Dann einem $\epsilon$ oder zwei $\delta$'s.
			\p\item Dann vier Zeichen aus A.
		\end{itemize}
		\pitem $L_3 = B \cdot B \cdot \{\epsilon, \delta\delta\} \cdot A \cdot A \cdot A \cdot A$.
	\end{itemize}
\end{frame}

\begin{frame}{Produkt von Sprachen}	
	\begin{taskblock}{Übung zu Produkt von formalen Sprachen}
		Sei $A$ ein beliebiges Alphabet und $M := \{L : L $ ist formale Sprache über $A \} \pause = 2^A$. \pause Produkt von Sprachen lässt sich auch als Abbildung bzw. Verknüpfung $\cdot : M \times M \rightarrow M$ darstellen.
		
		Zeige: 
		\begin{itemize}
			\pitem Die Verknüpfung $\cdot$ ist assoziativ.
			\pitem Es gibt (mindestens) ein Element $e \in M$, sodass für alle $x \in M$ gilt: $x \cdot e = e \cdot x = x$. (Neutrales Element)
			\pitem Es gibt ein Element $o \in M$, sodass für alle $x \in M$ gilt: $x \cdot o = o = o \cdot x$. (Absorbierendes Element)
		\end{itemize}
	\end{taskblock}
\end{frame}

\begin{frame}{Produkt von Sprachen}
	\pause 
	
	Seien $L_1, L_2, L_3 \in M$.
	
	\begin{itemize}
		\item Die Verknüpfung $\cdot$ ist assoziativ:
		\begin{itemize}
			\pitem $(L_1 \cdot L_2) \cdot L_3 \pause = (\{w_1\cdot w_2 : w_1 \in L_1, w_2 \in L_2\}) \cdot L_3 \pause = \{w_1w_2w_3 : w_1 \in L_1, w_2 \in L_2, w_3 \in L_3\} \pause = L_1 \cdot (\{w_2w_3 : w_2 \in L_2, w_3 \in L_3\}) \pause = L_1 \cdot (L_2 \cdot L_3)$.
		\end{itemize}
	
		\pitem Es gibt (mindestens) ein Element $e \in M$, sodass für alle $x \in M$ gilt: $x \cdot e = e \cdot x = x$. (neutrales Element)
		\begin{itemize}
			\pitem $e := \{\epsilon\}$.
			\pitem $L_1 \cdot \{\epsilon\} \pause = L_1 \pause = \{\epsilon\} \cdot L_1$
		\end{itemize}
	
		\pitem Es gibt ein Element $o \in M$, sodass für alle $x \in M$ gilt: $x \cdot o = o = o \cdot x$. (Absorbierendes Element)
		\begin{itemize}
			\pitem $o := \emptyset$
			\pitem $L_1 \cdot \emptyset = \emptyset = \emptyset \cdot L_1$
		\end{itemize}
	\end{itemize}

	$(M, \cdot)$ ist damit trotzdem keine Gruppe\p , denn es existieren keine Invers-Element.
\end{frame}

\begin{frame}{Potenz von Sprachen}
	
	\begin{block}{Potenz von Sprachen}
		Potenz von formellen Sprachen ist wie folgt definiert:
		\begin{itemize}
			\pitem $L^0 := \{\epsilon\}$
			\pitem $L^{i+1} := L^i \cdot L$ für $i \in \N_0$.
		\end{itemize}
	\end{block}

	\begin{itemize}
		\pitem $L_1 := \{a\}$.
		\begin{itemize}
			\pitem $L_1^0 = \{\epsilon\}$. \pause $L_1^1 = \{\epsilon\} \cdot L_1 = L_1$.
			\pitem $L_1^2 = (\{\epsilon\} \cdot L_1) \cdot L_1 \pause = \{aa\}$.
		\end{itemize}
		\pitem $L_2 := \{ab\}^3\{c\}^4$
		\begin{itemize}
			\pitem $L_2^0 = \{\epsilon\}, L_2^1 = ...$
			\pitem $L_2^2 \pause = (\{ab\}^3\{c\}^4)^2 \pause = (\{ab\}^3\{cccc\})^2 \pause = \{abababcccc\}^2 \pause = \{abababccccabababcccc\}$.
		\end{itemize}
		\pitem $L_3 := (\{a\} \cup \{b\})^2 \pause = \{aa, ab, ba, bb\}$
	\end{itemize}

\end{frame}

\begin{frame}{Konkatenationsabschluss bei formalen Sprachen}
	\p
	\begin{block}{Konkatenationsabschluss}
		Zu einer formalen Sprache $L$ ist der Konkatenationsabschluss $L^*$ definiert als \pause $L^* := \bigcup\limits_{i \in \N_0} L^i$.
	\end{block}
	\p
	\begin{block}{$\epsilon$-freie Konkatenationsabschluss}
		Zu einer formalen Sprache $L$ ist der $\epsilon$-freie Konkatenationsabschluss $L^+$ definiert als \pause $L^+ := \bigcup\limits_{i \in \N_+} L^i$.
	\end{block}

	\begin{itemize}
		\pitem Warum gilt $\epsilon \notin L^+$ bei formalen Sprache $L \subseteq A^* \backslash \{\epsilon\}$?
		\pitem $L := \{a, b, c\}.  L^* \pause = \{\epsilon, a, aa, ab, ac, aaa, aab, \dots, b, ba, bb, \dots \}$
		\pitem $L := \{aa, bc\}.  L^* \pause = \{\epsilon, aa, bc, aa\cdot aa, aa\cdot bc, bc \cdot aa, bc \cdot bc, aa \cdot aa \cdot aa, \dots \}$
	\end{itemize}
\end{frame}

\begin{frame}{Übung zu Konkatenationsabschluss}
	Sei $A := \{a, b\}, B := \{A, B, C, D, E, F\}$.
	\begin{itemize}
		\pitem Sprache $L_1 \subseteq A^*$, die das Teilwort $ab$ nicht enthält? \pause $L_1 = \{b\}^*\{a\}^*$.
		\pitem Sprache $L_2 \subseteq B^*$, die alle erlaubten Java Variablennamen enthält.
		\begin{itemize}
			\pitem $B := \{\_,a,b,...,z,A,B,...,Z\}$
			\pitem $C := B \cup \Z_9$
			\pitem $L_2 \subseteq C \pause = (B \cdot C^*) \backslash \{if, class, while, ...\}$
		\end{itemize}
	\end{itemize}
\end{frame}

\begin{frame}{Übung zu Konkatenationsabschluss}
	\pause Sei $L := \{a\}^*\{b\}^*$.
	\begin{itemize}
		\pitem Was ist alles in $L$ drin?
		\begin{itemize}
			\pitem $aaabbabbaaabba$? \pause Nein.
			\pitem $aaabb$, $abb$, $aaabba$, $a$? \pause Ja, ja, nein, ja.
		\end{itemize}
		\pitem Was ist alles in $L^*$ drin?
		\begin{itemize}
			\pitem $aaabbabbaaabba$? \pause Ja.
			\pitem $aaabb$, $abbaaabba$? \pause Ja.
			\pitem $abb$, $aaabba$, $a$? \pause Ja.
			\pitem Alle Wörter aus $\{a,b\}^*$! \pause $\rightarrow L^* = \{a,b\}^*$.
		\end{itemize}
	\end{itemize}
\end{frame}

\begin{frame} {Übung 1}
	\begin{itemize}
		\item $L_1 = \{a^k|k \in \N\}$
		\item $L_2 = \{b^k|k \in \N\}$
		\item Was ist $L_1 \cdot L_2$? \pause $\{a^i b^j|i,j \in \N\}$ 
	\end{itemize}
\end{frame}

\begin{frame} {Übung 2}
	\begin{itemize}
		\item $L_1 = \{a\}*\{b\}*$
		\item $L_2 = \{a, b\}*$
		\item $L_3 = \{a\}* \cup \{b\}*$
		\item $L_4 = (\{a\} \cup \{b\})*$
		\item Welche Sprachen sind identisch? \pause $L_2$ und $L_4$
	\end{itemize}
\end{frame}

\begin{frame}{Übung 3}
	$L_1, L_2$ seien formale Sprachen.
	\begin{itemize}
		\pitem Wie sieht $L_1 \cdot L_2$ aus?
		\pitem Wie sieht $L_1^3$ aus?
		\pitem Wie sieht $L_1^2 \cdot L_2 \cdot L_2^0 \cdot L_1^*$ aus?
		\pitem Wie sieht $(L_1^*)^0 \cdot L_2^+$ aus?
	\end{itemize}	
\end{frame}

\section{Übersetzung und Kodierung}

\begin{frame}{Herführung zu Zahlendarstellungen}
	\pause Wir betrachten die Alphabete $A_{dez} := \Z_{10}, A_{bin} := \{0, 1\}, A_{oct} := \Z_8$.
	\begin{itemize}
		\pitem Was können wir daraus machen?
		\pitem $A_{dez}^* \supset \{42, 1337, 999\}$.
		\pitem $A_{bin}^* \supset \{101010, 10100111001, 1111100111\}$.
		\pitem $A_{oct}^* \supset \{52, 2471, 1747\}$.
		\pitem Wir suchen eine Möglichkeit, diese \markGreen{Zahlen} zu \markGreen{deuten}.
		\pitem Aber irgendwie so, dass $42_{\in A_{dez}} \stackrel{Deutung}{=} 101010_{\in A_{bin}} \stackrel{Deutung}{=} 52_{\in A_{oct}}$.
	\end{itemize}
\end{frame}

\subsection{Kodierung von Zahlen}

\begin{frame}{Definition von Zahlendarstellungen}
	\pause
	
	\begin{block}{$Num_k$}
		Einer Zeichenkette $Z_k$ aus Ziffern \p wird mit $Num_k$ eine eindeutige Zahl zugeordnet:
		
		\vspace{.2cm}
		
		\vspace{.2cm}
		
		 \p $Num_k(\epsilon) = 0$
		
		\vspace{.2cm}
		
		 \p $Num_k(wx) = k \cdot Num_k(w) + num_k(x)$ mit $w \in Z_k^*$ und $x \in Z_k$.
	\end{block}

	\pause
	
	\begin{block}{$num_k$}
		Einer einzelnen Ziffer $x \in Z_k$ aus einem Alphabet von Ziffern $Z_k$ wird mit $num_k(x)$ der Wert der Zahl zugewiesen.
	\end{block}

	\pause
	
	\begin{itemize}
		\item Wichtig: $Num_k(w) \neq num_k(w)$!
		\pitem Was ist: $num_{10}(3) \p = 3\p , num_{10}(7) \p = 7\p , num_{10}(11) = \p $ nicht definiert.
		\pitem Für Zahlen $\geq k$: Benutze $Num_k$!
	\end{itemize}
\end{frame}

\begin{frame}{Beispiel zu Zahlendarstellungen}
	$Num_k(\epsilon) = 0$.
	
	$Num_k(wx) = k \cdot Num_k(w) + num_k(x)$ mit $w \in Z_k^*$ und $x \in Z_k$.
	
	\vspace{.3cm}
	
	\p Was ist $Num_{10}(123)$?
	\begin{itemize}
		\pitem $Num_{10}(123) \p = 10 \cdot Num_{10} (12) + num_{10}(3) \p = 10 \cdot ( Num_{10} (1) + num_{10}(2)) + num_{10}(3) \p = 10 \cdot ( num_{10}(1) + 10 \cdot num_{10}(2) ) + num_{10}(3) \p = 10 \cdot ( 1 + 10 \cdot 2 ) + 3 \p = 123$.
	\end{itemize}
	\p Yay?
	
	\p Was ist der dezimale Zahlenwert der Binärzahl 1010? \p Diesmal Basis $k = 2$.
	\begin{itemize}
		\pitem $Num_{2}(1010) \p = 2 \cdot Num_2(101) + num_2(0) \p = 2 \cdot (2 \cdot Num_2(10) + num_2(1) + num_2(0) \p = 2 \cdot (2 \cdot (2 \cdot Num_2(1) + num_2(0) ) + num_2(1) ) + num_2(0) \p = 2 \cdot (2 \cdot (2 \cdot num_2(1) + num_2(0) ) + num_2(1) ) + num_2(0) \p = 2 \cdot ( 2 \cdot (2 \cdot 1 + 0) + 1) + 0) = 10$.
	\end{itemize}
	\p Yay!
\end{frame}

\begin{frame}{Aufgaben zu Zahlendarstellungen}
	$Num_k(\epsilon) = 0$.
	
	$Num_k(wx) = k \cdot Num_k(w) + num_k(x)$ mit $w \in Z_k^*$ und $x \in Z_k$.
	
	\begin{taskblock}{Übungen zu Zahlendarstellungen}
		Berechne den numerischen Wert der folgenden Zahlen anderer Zahlensysteme nach dem vorgestellten Schema:
		\begin{itemize}
			\item $Num_8(345)$.
			\item $Num_2(11001)$.
			\item $Num_2(1000)$.
			\item $Num_4(123)$.
			\item $Num_{16}(4DF)$. (Zusatz)
		\end{itemize}
	\end{taskblock}

	Anmerkung: Hexadezimalzahlen sind zur Basis $16$ und verwenden als Ziffern (in aufsteigender Reihenfolge: $1, 2, 3, 4, 5, 6, 7, 8, 9, A, B, C, D, E, F$.
\end{frame}

\begin{frame}{Aufgaben zu Zahlendarstellungen}
	\pause Lösungen:
	\begin{itemize}
		\pitem $Num_8(345) \p = 229$.
		\pitem $Num_2(11001) \p = 25$.
		\pitem $Num_2(1000) \p = 8$.
		\pitem $Num_4(123) \p = 27$.
		\pitem $Num_{16}(4DF) \p = 1247$. 
	\end{itemize}
\end{frame}

\begin{frame}{Einfachere Umrechnung von Zahlendarstellungen}
	Es gilt: $2(2(2(2(2 \cdot 1 + 0)+1)+0)+1)+0 \p = 2^5 \cdot 1 + 2^4 \cdot 0 + 2^3 \cdot 1 + 2^2 \cdot 0 + 2^1 \cdot 1 + 2^0 \cdot 0$.
	
	\p Daher, einfachere Rechenweise: $Num_k(w) = k^0 \cdot w(0) + k^1 \cdot w(1) + k^2 \cdot w(2) + ...$.
	
	\p Was sind folgende Zahlen in Dezimal im Kopf gerechnet?
	
	\begin{itemize}
		\pitem $Num_2(10101) \p = 21$.
		\pitem $Num_2(11101) \p = 29$.
		\pitem $Num_2(1111111111) \p = 1023$.
	\end{itemize}
	
\end{frame}
		
\begin{frame}{Einfachere Umrechnung von Zahlendarstellungen}
	$Num_k(w) = k^0 \cdot w(0) + k^1 \cdot w(1) + k^2 \cdot w(2) + ...$.
	
	\p Was sind folgende Zahlen in Dezimal im Kopf gerechnet?
	
	\begin{itemize}
		\pitem $Num_{16}(A1) \p = 161$.
		\pitem $Num_{16}(BC) \p = 188$.
		\pitem $Num_{16}(14) \p = 20$.
	\end{itemize}
	
\end{frame}


\begin{frame}
	\includegraphics[width=\linewidth]{../images/thatsall.png}
\end{frame}


\end{document}
