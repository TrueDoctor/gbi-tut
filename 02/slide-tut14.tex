\def\tutdate{30.10.2019}

\documentclass[handout]{beamer}
\usepackage{../templates/beamerthemekit}
%\usepackage{enumitem}
\input{../assets/dependencies}
\input{../assets/definitions}
\title[Grundbegriffe der Informatik]{GBI\\Tutorium 8}
\date{\tutdate}
\subtitle{\tutTitle}
\author{Dennis Kobert dennis@kobert.dev}

\institute{}

\titleimage{bg}
%\titleimage{bg-advent}

%\include{assets/multislider}

\setbeamercovered{invisible}

%\usepackage[citestyle=authoryear,bibstyle=numeric,hyperref,backend=biber]{biblatex}
%\addbibresource{templates/example.bib}
%\bibhang1em

	
\def\tutTitle{Wörter und Formale Sprachen}
\begin{document}

\selectlanguage{ngerman}

%title page
\begin{frame}
	\titlepage
\end{frame}

\section{Hinweise}

\begin{frame}{Hinweise zur Abgabe}
	\begin{itemize}
		\item Blätter vollständig nutzen, nicht ein Blatt pro Aufgabe
		\item Aufgabenblatt nicht mit abgeben
		\item Nur antworten, was gefragt wurde (keine falschen oder nicht gefragten Lösungswege)
		\item "`geben Sie an"' erfordert keinen Beweis
		\item Lieber sprachlich erklären, wenn man Probleme mit formalen Beweisen hat (falls nicht explizit einer gefordert ist)
		\item Genau schreiben, warum die Behauptung aus dem folgt, was ihr gerade bewiesen habt
		\item Unterscheidet Element und Teilmenge: $x \in A$ oder $\{x\} \subset A$
		\item Vereinigung ist $\cup$ und nicht $+$
	\end{itemize}
\end{frame}

\begin{frame}{Typische Fehler}
	\begin{itemize}
		\item Sprachlich: A oder B ist die leere Menge
		\item Formal: $A = \emptyset \lor B = \emptyset$ Nicht: $A \lor B = \emptyset$
		\item Leere Menge ist $\emptyset $ oder $\{\}$, nicht 0
	\end{itemize}
\end{frame}

\section{Mengen}

\begin{frame}{Mengen}
	Seien $A=\{1,2,3,4\}, B=\{3,4,5,6,7\}, C=\{5,6,7\}$. Bestimme
	\begin{itemize}
		\item $A \cup B$
		\item $A \cup C$
		\item $A \cap B$
		\item $A \cap C$
	\end{itemize}
	sowie die Kardinalität dieser Mengen.
\end{frame}
\begin{frame}{Mengen}
	Lösung:
	\begin{itemize}
		\item $A \cup B = \{1,2,3,4,5,6,7,\}$ und $\mid A \cup B\mid = 7$
		\pause
		\item $A \cup B = \{1,2,3,4,5,6,7,\}$ und $\mid A \cup C\mid = 7$
		\pause
		\item $A \cap B = \{3,4\}$ und $\mid A \cap B \mid = 2$
		\pause
		\item $A \cap C = \emptyset$ und $\mid\emptyset\mid = 0$
	\end{itemize}
\end{frame}

\begin{frame}{Mengen}
	\begin{block}{Aufgabe aus WS16/17}
		Es seinen A,B und C Mengen. Beweisen oder widerlegen Sie:
		$A \backslash (B \backslash C) = (A \backslash B)\backslash C$
	\end{block}
\end{frame}
\begin{frame}{Mengen}
	\begin{block}{Aufgabe aus WS16/17}
		Es seinen A,B und C Mengen. Beweisen oder widerlegen Sie:
		$A \backslash (B \backslash C) = (A \backslash B)\backslash C$
	\end{block}
	\begin{block}{Widerlegung durch Gegenbeispiel}
		Seien $A = \{1,2,3\}, B = \{3,4,5\}$ und $C = \{3\}$.
		Dann ist $\{1,2,3\} \backslash (\{3,4,5\} \backslash \{3\}) = \{1,2,3\} \backslash \{4,5\} = \{1,2,3\} \neq$
		$\{1,2\} = \{1,2\} \backslash \{3\} = (\{1,2,3\} \backslash \{3,4,5\}) \backslash \{3\}$
	\end{block}
\end{frame}
\begin{frame}{Mengenäquivalenz}
	\begin{block}{Zeigen Sie}
		$A \cup (B \cap C) = (A \cup B) \cap (A \cup C)$
		\begin{itemize}
			\pause\item ``$\subseteq$'': Sei $x \in A \cup (B\cap C).$ Dann ist $x \in A$ oder $x \in (B \cap C)$
			\begin{itemize}
				\pause
				\item Falls $x\in A$, dann gilt auch $x\in (A\cup B)$ und $x\in (A\cup C)$. Also
				insbesondere $x\in(A\cup B)\cap(A\cup C)$.
				\pause
				\item Falls $x\in(B\cap C)$, dann gilt auch $x\in (A\cup B)$ und $x\in (B\cup C)$. Also
				insbesondere $x\in(A\cup B)\cap(A\cup C)$.
				\pause
			\end{itemize}
			\item ``$\supseteq$'': $x\in(A\cup B)\cap(A\cup C)$. Dann liegt x in $(A\cup B)$ und $(A\cup C)$.
			Also liegt x in A oder in (B und C). Folglich gilt 
			$x\in A\cup(B\cap C)$.
			\pause\item Insgesamt ist also $A\cup(B\cap C) = (A\cup B)\cap(A\cup C)$.
		\end{itemize}
	\end{block}
\end{frame}

\begin{frame}{Mengen}
	\begin{block}{Aufgabe aus WS16/17}
		Es sei M eine Menge und es seien $A\subseteq M$ und $B\subseteq M$. Beweisen Sie:
		$M\backslash(A\cap B) = (M\backslash A)\cup(M\backslash B)$
		\begin{itemize}
			\pause \item ``$\subseteq$'': Sei $x\in M\backslash(A\cap B)$. Dann ist $x\in M$ und $x\notin A$ oder $x\notin B$.
			\\Somit ist
			\begin{itemize}
				\item $x\in M$ und $x\notin A$ oder
				\item $x\in M$ und $x\notin B$.
			\end{itemize}
			Also ist $x\in(M\backslash A)$ oder $(M\backslash B)$, folglich also
			$x\in(M\backslash A)\cup(M\backslash B)$.

			\pause\item ``$\supseteq$'': Sei $x\in(M\backslash A)\cup(M\backslash B)$. Dann ist $x\in(M\backslash A)$ oder\\
			$x\in(M\backslash B)$. Somit ist
			\begin{itemize}
				\item $x\in M$ und $x\notin A$ oder
				\item $x\in M$ und $x\notin B$.
			\end{itemize}
			Also ist $x\in M$, und $x\notin A$ oder $x\notin B$. Dann ist $x\in M$ und \\
			$x\notin(A\cap B)$, folglich also $x\in M\backslash(A\cap B)$.
		\end{itemize}
	\end{block}
\end{frame}

\section{Wörter}

\begin{frame}{Wörter}
	\begin{block}{Konkatenation}
		Durch Konkatenation werden einzelne Buchstaben aus einem Alphabet miteinander verbunden.
	\end{block}

	\begin{itemize}
		\pitem Symbol: $\cdot$\pause , also zwei Buchstaben $a$ und $b$ miteinander konkateniert: $a \cdot b$.
		\pitem Nicht kommutativ : $a \cdot b \neq b \cdot a$
		\pitem Aber assoziativ : $(a \cdot b) \cdot c = a \cdot (b \cdot c)$
		\pitem Kurzschreibweise : Ohne Punkte , also $a \cdot b = ab$
	\end{itemize}
\end{frame}

\begin{frame}{Wörter}
	\begin{block}{Wörter: Intuitivere Definition}
		Ein Wort $w$ entsteht durch die Konkatenation durch Buchstaben aus einem Alphabet.
	\end{block}

	$\rightarrow$ Also Abfolge von Zeichen über ein Alphabet A.

	\pause Sei $A := \{a, b, c\}$.

	\begin{itemize}
		\pitem Mögliche Worte: \pause $w_1 := a \cdot b$\pause , $w_2 = b \cdot c \cdot c$\pause , $w_3 = a \cdot c \cdot c \cdot b \cdot a$.
		\pitem Keine möglichen Worte: \pause $d$.
		\end{itemize}
\end{frame}

\begin{frame}{Wörter}
	\begin{block}{Wörter: Abstraktere Definition}
		Ein Wort $w$  über dem Alphabet $A$  ist definiert als surjektive Abbildung  $w : \Z_n \rightarrow A$. Dabei heißt $n$ die Länge $|w|$ des Wortes.
	\end{block}

	\begin{itemize}
		\pitem $\Z_n$  $ = \{i \in \N : 0 \leq i < n \}$
		
		\pause $\Z_3  = \{0, 1, 2\},  \Z_2 = \{0, 1\}, \Z_0 = \emptyset$.
		
		\pitem Länge oder Kardinalität eines Wortes:  $|w|$. \pause $|abcde|$ $= 5$.
		
		\pitem Wort $w = abdec$ als Relation aufgeschrieben: \pause $w = \{(0, a), (1, b), (2, d), (3, e), (4, c)\}$.  Also $w(0) = a, w(1) = b, w(2) = d, \dots$
		
		\pause Damit sieht man auch: $|w| = |\{(0, a), (1, b), (2, d), (3, e), (4, c)\}| = 5$.
	\end{itemize}
\end{frame}

\begin{frame}{Wörter}
	\begin{itemize}
		\item Wort der Kardinalität 0?
	\end{itemize}

	\pause

	\begin{block}{Das leere Wort}
		Das leere Wort $\varepsilon$ ist definiert ein Wort mit Kardinalität 0 , also mit 0 Zeichen.
	\end{block}

	\begin{itemize}
		\pitem Leere Wort wird interpretiert als ``nicht sichtbar'' und kann überall platziert werden\pause : $aabc = a\epsilon abc = a\epsilon\epsilon a\epsilon bc \epsilon$.
		\pitem $|\{\epsilon\}|$ \pause $ = 1$ , die Menge ist nicht leer! Das leere Wort ist nicht \emph{nichts}! (Vergleiche leere Menge)
		\pitem $|\epsilon| = 0$.
		\pitem Formale Definition:\pause $\epsilon : \emptyset \rightarrow \emptyset $
	\end{itemize}
\end{frame}

\begin{frame}{Mehr über Wörter}
	\begin{block}{$A^n$}
		Zu einem Alphabet $A$ ist $A^n$ definiert als die Menge aller Wörter der Länge $n$ über dem Alphabet $A$.
	\end{block}

	\begin{itemize}
		\pitem Nicht mit Mengenpotenz verwechseln!
		\pitem $A := \{a, b, c\}$\pause , $A^2 = \{aa, ab, ac, ba, bb, bc, ca, cb, cc\}$. \pause $A^1 = A, \pause A^0 = \{\epsilon\}$.
	\end{itemize}
	
	\pause Die Menge aller Wörter  \emph{beliebiger} Länge: \pause
	\begin{itemize}
		\item $A^* := \bigcup_{i \in \N_0} A_i$
		\pitem $A := \{a, b, c\}$ . $aa \in A^* , abcabcabc \in A^* , aaaa \in A^* , \epsilon \in A^*$.
	\end{itemize}
\end{frame}

\begin{frame}{Mehr über Wörter}
	\begin{block}{Wort Potenzen}
		Sich direkt wiederholende Teilworte kann man als Wortpotenz darstellen , daher $w_i^n = w_i \cdot w_i \cdots w_i$ (n $\times$ mal).
	\end{block}

	\begin{itemize}
		\pitem $a^4 = aaaa$\pause , $b^3 = bbb$\pause , $c^0 = \pause \epsilon$\pause , $d^1 = \pause d$.
		\pitem $a^3c^2b^6 \pause = aaaccbbbbbb$.
		\pitem $b \cdot a \cdot (n \cdot a)^2$ \pause $ = banana$.
		\end{itemize}
\end{frame}

\begin{frame}{Übung zu Wörter}
	Sei $A$ ein Alphabet.
	
	\begin{taskblock}{Übung zu Wörter}
		\begin{enumerate}
			\item Finde Abbildung $f: A^* \rightarrow A^*$, sodass für alle $w \in A^*$ gilt: \\\quad $2 \cdot |w| = |f(w)|$.
			\item Finde Abbildung $g: A^* \rightarrow A^*$, sodass für alle $w \in A^*$ gilt: \\\quad $|w| + 1 = |g(w)|$.
			\item Sind $f, g$ injektiv und/oder surjektiv?
		\end{enumerate}
	\end{taskblock}

	\pause
	
	\begin{enumerate}
		\item $f: A^* \rightarrow A^*, w \mapsto w \cdot w$.
		\pitem $g: A^* \rightarrow A^*, w \mapsto w \cdot x, x \in A$.
	\end{enumerate}
\end{frame}

\begin{frame}{Übung zu Wörter}
	\begin{enumerate}
		\item $f: A^* \rightarrow A^*, w \mapsto w \cdot w$.
		\begin{itemize}
			\pitem $f$ ist injektiv\pause , denn jedes $w$ aus der Bildmenge wird von maximal einem Wort abgebildet.
			\pitem $f$ ist nicht surjektiv\pause , denn z.B. bildet nichts auf $x \in A$ ab (oder auf andere Wörter mit ungerader Anzahl an Buchstaben).
		\end{itemize}
		\pitem $g: A^* \rightarrow A^*, w \mapsto w \cdot x, x \in A$.
		\begin{itemize}
			\pitem $g$ ist injektiv.
			\pitem $g$ ist nicht surjektiv\pause , denn z.B. bildet nichts auf $\epsilon$ ab.
		\end{itemize}
	\end{enumerate}
\end{frame}

\section{Formale Sprachen}

\begin{frame}{Formale Sprache}
	\begin{itemize}
		\pitem Was war nochmal $A^*$? Menge aller Wörter \emph{beliebiger} Länge über Alphabet $A$.
	\end{itemize}

	\pause
	
	\begin{block}{Formale Sprache}
		Eine Formale Sprache $L$ über einem Alphabet $A$ ist eine Teilmenge $L \subseteq A^*$.
	\end{block}

	\begin{itemize}
		\pitem Zufälliges Beispiel: \pause $A := \{b, n, a\}$.
		\begin{itemize}
			\pitem $L_1 := \{ban, baan, nba, aa\}$ ist eine mögliche formale Sprache über $A$.
			\pitem $L_2 := \{banana, bananana, banananana, ...\}$ \pause $ = \{w : w = bana(na)^k, k \in \N\}$ auch.
			\pitem $L_3 := \{ban, baan, baaan, ...\}$ auch. \pause Andere Schreibweise? \pause \\ $ L_3 = \{w : w = ba^kn, k \in \N \}$
		\end{itemize}
		\pitem Formale Sprachen sind also nicht zwangsweise endliche Mengen.
		\pitem Praktischeres Beispiel: $A := \{w : w $ ist ein ASCII Symbol $\}$.
		\pitem $L_4 := \{w : $ w ist korrekt kompilierendes Java-Programm $\}$ 
	\end{itemize}
\end{frame}

\begin{frame}{Übung zu formalen Sprachen}
	$A := \{a, b\}$
	
	\begin{itemize}
		\pitem Sprache $L$ aller Wörter über $A$, die nicht das Teilwort $ab$ enthalten?
		\begin{itemize}
			\pitem Was passiert wenn ein solches Wort ein $a$ enthält? \pause Dann keine $b$'s mehr!
			\pitem $L = \{w_1 \cdot w_2 : w_1 \in \{b\}^*$ und $w_2 \in \{a\}^* \}$
			\pitem Andere Möglichkeit\pause : Suche Wörter mit $ab$ und nehme diese Weg.
			\pitem $L = \{a, b\}^*\pause \backslash\{ w_1 \cdot ab \cdot w_2 : w_1, w_2 \in \{a,b\}^* \}$
		\end{itemize}
	\end{itemize}
\end{frame}

\begin{frame}{Übung zu formalen Sprachen}
	
	Sei $A := \{a, b\}, B := \{0, 1\}$.
	
	\begin{taskblock}{Aufgabe zu formalen Sprachen}
		\begin{enumerate}
			\item Sprache $L_1 \subseteq A^*$ von Wörtern, die mindestens drei $b$'s enthalten.
			\item Sprache $L_2 \subseteq A^*$ von Wörtern, die gerade Zahl von $a$'s enthält.
			\item Sprache $L_3 \subseteq B^*$ von Wörtern, die, interpretiert als Binärzahl eine gerade Zahl sind.
		\end{enumerate}
	\end{taskblock}

	\pause
	
	\begin{enumerate}
		\item $L_1 = \{w = w_1  b  w_2  b  w_3 b w_4 : w_1,w_2,w_3,w_4 \in A^* \}$
		\pitem $L_2 = \{w = (w_1 a w_2 a w_3)^* : w_1,w_2,w_3 \in \{b\}^* \}$ \pause (Ist da $\epsilon$ drin?)
		\pitem $L_3 = \{w = w \cdot 0 : w \in B^* \}$
	\end{enumerate}
\end{frame}

\section{Aussagenlogik}

\begin{frame}{Aussagenlogik}
	\begin{itemize}
		\pitem Das wars erst mal zu formalen Sprachen.
		\pitem Heute ist Donnerstag.
		\pitem Die Menge aller HM-Übungsblätter dieser Welt ist disjunkt zur Menge aller einfachen Übungsblätter dieser Welt.
	\end{itemize}

	\pause
	
	Das sind alles Aussagen. Aussagen sind entweder \emph{wahr} oder \emph{falsch}.
\end{frame}

\begin{frame}{Aussagenlogik}
	
	Wir kapseln Aussagen und verwendet Variablen dafür. 
	
	\begin{itemize}
		\pitem $A := $ ``Die Straße ist nass.''
		\pitem $B := $ ``Es regnet.''
	\end{itemize}

	\pause Aussagen lassen sich verknüpfen:
	
	\begin{itemize}
		\pitem \markGreen{Logisches Und:} $A \land B = A$ und $B = $ Die Straße ist nass und es regnet.
		\pitem \markGreen{Logisches Oder:}  $A \lor B  = A$ oder $B  = $ Die Straße ist nass oder es regnet . Es kann auch beides wahr sein.
		\pitem \markGreen{Negierung:}  $\lnot A  = $ nicht $A  = $ Die Straße ist nicht nass.
		\pitem \markGreen{Implikation:}  $A \rightarrow B  = $ Aus $A$ folgt $B  = $ Wenn die Straße nass ist, dann regnet es.
		\pitem \markGreen{Äquivalenz:}pause $A \leftrightarrow B  = A$ und $B$ sind äquivalent $ = $ Die Straße ist \emph{genau dann} nass, \emph{wenn} es regnet.
		\begin{itemize}
			\pitem $A \leftrightarrow B = (A \rightarrow B) \land (B \rightarrow A)$  , also die Straße ist nass wenn es regnet \emph{und} es regnet wenn die Straße nass ist.
		\end{itemize}
	\end{itemize}

\end{frame}

\begin{frame}{Übung zu Aussagenlogik}
	
	\begin{itemize}
		\item $A := $ ``Die Straße ist nass.''
		\item $B := $ ``Es regnet.''
		\item $C := $ ``$\pi$ ist gleich $3$.''
	\end{itemize}

	\begin{itemize}
		\pitem Was ist $B \rightarrow C$?   ``Wenn es regnet, ist $\pi$ gleich $3$.''
	\end{itemize}

	\pause
	
	% Aus Skriptum, Kapitel 5.3 Boolesche Funktionen
	\begin{center}
		\begin{tabular}{c|c||c|c|c|c}%*{2}{>{$}c<{$}}|*{4}{>{$}c<{$}}
			\hline
			$x_1$ & $x_2$ & $\lnot x_1$ & $x_1 \land x_2$ & $x_1 \lor x_2$ & $x_1 \rightarrow x_2$ \\
			\hline
			\F & \F & \W & \F & \F & \W \\
			\F & \W & \W & \F & \W & \W \\
			\W & \F & \F & \F & \W & \F \\
			\W & \W & \F & \W & \W & \W \\
			\hline
		\end{tabular}
	\end{center}
	
\end{frame}

\begin{frame}{Syntax der Aussagenlogik}
	\pause
	Menge der Aussagevariablen:
	
	\pause\quad $Var_{AL} \pause \subseteq \{P_i : i \in \N_0\}$ \pause oder $\{P, Q, R, S, \dots\}$
	
	\pause Alphabet der Aussagenlogik:
	
	\pause\quad $A_{AL} = \{(, ), \lnot, \land, \lor, \rightarrow, \leftrightarrow\} \cup Var_{AL}$
\end{frame}

\begin{frame}{Boolesche Funktionen}
\begin{block}{Boolesche Funktionen}
	Eine boolsche Funktion ist eine Abbildung \pause der Form $f: \B^n \rightarrow \B$ \pause mit $\B = \{w, f\}$.
\end{block}

Typische Boolsche Funktionen\pause : $b_\lnot (x) \pause = \lnot x$\pause , $b_\lor (x_1, x_2) \pause = x_1 \lor x_2$ \dots
\end{frame}

\begin{frame}{Interpretationen}
	\begin{block}{Interpretation}
		\pause Eine Interpretation ist eine Abbildung $I : V \rightarrow \B$\pause , die einer Variablenmenge eine ``Interpretation''\pause , also wahr oder falsch zuordnet.
	\end{block}

\pause Weiter legt man $val_I(F)$ als Auswertung einer aussagenlogischer Formel $F$ fest.
\pause
	\newcommand{\val}{val}
	\begin{align*}
	\val_I(X)         &= I(X) \\
	\val_I(\lnot G)   &= b_{\lnot}(\val_I(G)) \\
	\val_I(G \land H) &= b_{\land}(\val_I(G), \val_I(H)) \\
	\val_I(G \lor H)  &= b_{\lor}(\val_I(G)  \val_I(H)) \\
	\val_I(G \rightarrow H)&= b_{\rightarrow}(\val_I(G), \val_I(H))
	\end{align*}
\end{frame}

\begin{frame}{Übung zu Interpretationen}
	
	
	\begin{itemize}
		\item Wie viele Interpretationen gibt es bei k = 1, 2, 3 Variablen?
		\item Wie viele Interpretationen gibt es bei k+1 Variablen im Vergleich zu k Variablen?
	\end{itemize}
	
\end{frame}	

\begin{frame}{Übung zur Aussagenlogik}
	\pause Sei $A := \W, B := \W, C := \F$.
	
	\begin{itemize}
		\pitem Ist $(A \land B) \lor \lnot C$ wahr oder falsch? \pause $(A \land B) \lor \lnot C \pause = (\W \land \W) \lor \lnot \F \pause = \W \lor \lnot \F = \pause \W \lor \W \pause = \W$\pause , die Aussage ist also wahr.
		\pitem Ist $\lnot (A \lor A)$ wahr oder falsch? \pause Falsch! \pause Wann ist $\lnot (A \lor A)$ im allgemeinen wahr? \pause Genau dann, wenn $\lnot A$ wahr ist.
	\end{itemize}

	\pause

	\begin{block}{Äquivalenz von Aussagen}
		Erinnerung: \pause $A \leftrightarrow B$ heißt: \pause $(A \rightarrow B) \land (B \rightarrow A)$. 
		
		\pause Wenn zwei Aussagen äquivalent sind, sind ihre Wahrheitswerte immer gleich\pause , wenn die Wahrheitswerte, von denen sie abhängen, gleich sind. 
		
		\pause Mann sagt und schreibt dann: \pause $A$ ist \emph{genau dann} wahr, \emph{wenn} $B$ wahr ist.
	\end{block}

	\begin{itemize}
		\pitem $\lnot (A \lor A)$ ist genau dann wahr\pause , wenn $\lnot A$ wahr ist\pause , also gilt:  $\lnot (A \lor A) \pause \leftrightarrow \lnot A$. 
	\end{itemize}
\end{frame}

\begin{frame}{Mehr zu Äquivalenz}
\pause
	\begin{block}{Alternative Definition zu Äquivalenz}
		Zwei Formeln G und H heißen äquivalent, wenn für jede Interpretation gilt $val_I(G) = val_I(H)$.
	\end{block}\pause

	Vorher Äquivalenz von Formeln unter gegebener Interpretation\pause , diesmal Äquivalenz von Formeln unter beliebiger Interpretation.\pause

	\textbf{Bemerkung}\\
	\begin{itemize}
		\pitem Man schreibt $G \equiv  H$
		\pitem $\mathbb{B}^V \rightarrow \mathbb{B}: I \mapsto val_I(G)$
	\end{itemize}\pause
	\textbf{Beispiele}\\\pause
	$(\lnot(\lnot P))$ ist äquivalent zu $P$\\\pause
	$(\lnot(P\land Q))$ ist äquivalent zu $((\lnot P) \lor (\lnot Q))$
\end{frame}

\begin{frame}{Beispiele zu Äquivalenz}
	\begin{itemize}
		\pitem Ein Wort $w$ hat die Länge $n$ $\leftrightarrow |w| = n$.
		\pitem Die Vereinigung zweier Mengen $A$ und $B$ hat die Kardinalität $|A| + |B|$ \pause $\leftrightarrow$ $A \cap B = \emptyset$ \pause $\leftrightarrow$ $A$ und $B$ sind disjunkt.
		\pitem $p$ ist eine rationale Zahl \pause $\leftrightarrow$ $p$ lässt sich darstellen als $p = \frac{a}{b}, a\in \Z, b \in \N$ \pause $\leftrightarrow$ $p \in \Q$.
	\end{itemize}	
\end{frame}

\begin{frame}{Wahrheitstabellen}
	\begin{itemize}
		\item $(((P \land Q) \lor Q) \rightarrow (P \land \lnot Q))$
	\end{itemize}

	\begin{center}
		\begin{tabular}{c|c||c|c|c|c}%*{2}{>{$}c<{$}}|*{4}{>{$}c<{$}}
			\hline
				$P$ & $Q$ & $(P \land Q)$ & $\lor Q$ & $\rightarrow$ & $(P \land \lnot Q)$ \\\hline
				
				\visible<1->{\W} & \visible<1->{\W} & \visible<2->{\W} & \visible<6->{\W} & \visible<14->{\F} & \visible<10->{\F} \\\hline
				
				\visible<1->{\W} & \visible<1->{\F} & \visible<3->{\F} & \visible<7->{\F} & \visible<15->{\W} & \visible<11->{\W} \\\hline
				
				\visible<1->{\F} & \visible<1->{\W} & \visible<4->{\F} & \visible<8->{\W} & \visible<16->{\F} & \visible<12->{\F} \\\hline
				
				\visible<1->{\F} & \visible<1->{\F} & \visible<5->{\F} & \visible<9->{\F} & \visible<17->{\W} & \visible<13->{\F} \\\hline
				
		\end{tabular}
	\end{center}
\end{frame}

\begin{frame}{Übungen zu Aussagenlogik}
	\begin{taskblock}{Übungen zu Aussagenlogik}
		\begin{itemize}
			\item Schreibe Wahrheitstabellen zu den Formeln um den Wahrheitsgehalt festzustellen.
				\item $\lnot(P \land Q) \land \lnot (Q \land P)$
				\item $(P \land Q \land R) \leftrightarrow (\lnot P \lor Q)$
				\item $(A\land(B\lor C))\leftrightarrow ((A\land B)\lor(A\land C))$
			\item Welche dieser Aussagen sind wahr?
				\item $\lnot (P \land Q) = \lnot P \lor \lnot Q$
				\item $P \land P = P \lor P$
				\item $(P \lor Q) \land R = (P \land R) \lor (Q \land R)$
		\end{itemize}
	\end{taskblock}
\end{frame}

\begin{frame} {Wahrheitstabellen}
	\begin{center}
		\begin{tabular}{|c|c|c|c|c|c|c|}
			\hline
			$A$&$B$& $\lnot A$& $A \land B$ & $A\lor B$ &$A\rightarrow B$ &$A \leftrightarrow B$\\
			\hline
			w&w&f&w&w&w&w\\
			w&f&f&f&w&f&f\\
			f&w&w&f&w&w&f\\
			f&f&w&f&f&w&w\\
			\hline
		\end{tabular}
	\end{center}

	\begin{taskblock}{Aufgabe}
		Finde einen logischen Ausdruck in A und B unter Verwendung von $\land, \lor$ und $\lnot$, der die Aussage ``Entweder A oder B'' repräsentiert	
	\end{taskblock}
\end{frame}

\begin{frame}{Wahrheitstabellen}
	
	\begin{taskblock}{Aufgabe}
		Finde einen logischen Ausdruck in A und B unter Verwendung von $\land, \lor$ und $\lnot$, der die Aussage ``Entweder A oder B'' repräsentiert	
	\end{taskblock}

	\textbf{Lösung}
	\begin{center}
		\begin{tabular}{|c|c|c|c|c|}
			\hline
			$A$&$B$& $A \land \lnot B$& $\lnot A \land B$ & $(A \land \lnot B) \lor (\lnot A \land B) $\\
			\hline
			w&w&f&f&f\\
			w&f&w&f&w\\
			f&w&f&w&w\\
			f&f&f&f&f\\
			\hline
		\end{tabular}
	\end{center}
\end{frame}


\begin{frame}{Weitere Begriffe}\pause
	\begin{block}{Tautologie}\pause
		Die Formel $G$ ist eine Tautologie (oder allgemeingültig)\pause , wenn $G$ für alle Interpretationen wahr ist.
	\end{block}\pause
	\begin{block}{Erfüllbarkeit}\pause
		Eine Formel $G$ ist erfüllbar\pause , wenn sie für mindestens eine Interpretation wahr ist.
	\end{block}
	\pause
	\begin{block}{Lemma}
		Wenn $G\equiv H$ ist, dann ist $G \leftrightarrow H$ eine Tautologie.
	\end{block}
\end{frame}

\begin{frame} {Übung zu Tautologien}
Sind das Tautologien?
\begin{itemize}
	\item $(G \rightarrow (H \rightarrow K)) \rightarrow ((G \rightarrow H) \rightarrow (G \rightarrow K))$ \pause \hspace{0.3cm} 
	\item $(\lnot P \rightarrow Q) \land R \lor P$ \pause \hspace{0.3cm} 
	\item $G \rightarrow (H \rightarrow G)$ \pause \hspace{0.3cm} 
	\item $(\lnot P \rightarrow \lnot Q) \rightarrow ((\lnot P \rightarrow Q) \rightarrow P)$ \pause \hspace{0.3cm} 
\end{itemize}
\end{frame}

\begin{frame} {Übung zu Tautologien Lösung}
Sind das Tautologien?
\begin{itemize}
	\item $(G \rightarrow (H \rightarrow K)) \rightarrow ((G \rightarrow H) \rightarrow (G \rightarrow K))$ \pause \hspace{0.3cm} Ja
	\item $(\lnot P \rightarrow Q) \land R \lor P$ \pause \hspace{0.3cm} Nein
	\item $G \rightarrow (H \rightarrow G)$ \pause \hspace{0.3cm} Ja
	\item $(\lnot P \rightarrow \lnot Q) \rightarrow ((\lnot P \rightarrow Q) \rightarrow P)$ \pause \hspace{0.3cm} Ja
\end{itemize}
\end{frame}


\begin{frame} {Übung zu Erfüllbarkeit}
	Sind die folgenden Ausdrücke erfüllbar?
	\begin{itemize}
		\item $ \lnot (A \lor \lnot A) $ \pause \hspace{0.3cm} 
		\item $(P \land \lnot Q) \lor (\lnot P \land R)$ \pause \hspace{0.3cm} 
		
	\end{itemize}
\end{frame}

\begin{frame} {Übung zu Erfüllbarkeit Lösung}
	Sind die folgenden Ausdrücke erfüllbar?
	\begin{itemize}
		\item $ \lnot (A \lor \lnot A) $ \pause \hspace{0.3cm} nein
		\item $(P \land \lnot Q) \lor (\lnot P \land R)$ \pause \hspace{0.3cm} Ja
		
	\end{itemize}
\end{frame}


\begin{frame}
	\includegraphics[width=\linewidth]{../images/thatsall.png}
\end{frame}


\end{document}
