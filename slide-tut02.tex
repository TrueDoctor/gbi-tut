\def\tutdate{2.11.2017}
\input{assets/slide-pre}

\section{Wiederholung und Übung}
\begin{frame}{Maß für Information}
	Die Zahl der verschiedenen mäglichen Nachrichten ist ein Maß für die\\
	Information einer Nachricht, wenn das Auftreten der Nachrichten\\
	 gleichverteilt ist.
	 \begin{itemize}
		 \item Logarithmus naturalis: natural units (nat)
		 \item Logarithmus zur Basis 10: Hartley (Hart)
		 \item Logarithmus dualis: Shannon (Sh)
	 \end{itemize}
\end{frame}

\begin{frame}{Mengen}
	Seien $A=\{1,2,3,4\}, B=\{3.4.5.6.7\}, C=\{5,6,7\}$. Bestimme
	\begin{itemize}
		\item $A \cup B$
		\item $A \cup C$
		\item $A \cap B$
		\item $A \cap C$
	\end{itemize}
	sowie die Kardinalität dieser Mengen.
\end{frame}
\begin{frame}{Mengen}
	Lösung:
	\begin{itemize}
		\item $A \cup B = \{1,2,3,4,5,6,7,\}$ und $\mid A \cup B\mid = 7$
		\pause
		\item $A \cup B = \{1,2,3,4,5,6,7,\}$ und $\mid A \cup C\mid = 7$
		\pause
		\item $A \cap B = \{3,4\}$ und $\mid A \cap B \mid = 2$
		\pause
		\item $A \cap C = \emptyset$ und $\mid\emptyset\mid = 0$
	\end{itemize}
\end{frame}

\begin{frame}{Mengen}
	\begin{block}{Aufgabe aus WS16/17}
		Es seinen A,B und C Mengen. Beweisen oder widerlegen Sie:
		$A \backslash (B \backslash C) = (A \backslash B)\backslash C$
	\end{block}
	\pause
	\begin{block}{Widerlegung durch Gegenbeispiel}
		Seien $A = \{1,2,3\}, B = \{3,4,5\}$ und $C = \{3\}$.
		Dann ist $\{1,2,3\} \backslash (\{3,4,5\} \backslash \{3\}) = \{1,2,3\} \backslash \{4,5\} = \{1,2,3\} \neq$
		$\{1,2\} = \{1,2\} \backslash \{3\} = (\{1,2,3\} \backslash \{3,4,5\}) \backslash \{3\}$
	\end{block}
\end{frame}
\begin{frame}{Mengenäquivalenz}
	\begin{block}{Zeigen Sie}
		$A \cup (B \cap C) = (A \cup B) \cap (A \cup C)$
		\begin{itemize}
			\pause\item ``$\subseteq$'': Sei $x \in A \cup (B\cap C).$ Dann ist $x \in A$ oder $x \in (B \cap C)$
			\begin{itemize}
				\pause
				\item Falls $x\in A$, dann gilt auch $x\in (A\cup B)$ und $x\in (A\cup C)$. Also
				insbesondere $x\in(A\cup B)\cap(A\cup C)$.
				\pause
				\item Falls $x\in(B\cap C)$, dann gilt auch $x\in (A\cup B)$ und $x\in (B\cup C)$. Also
				insbesondere $x\in(A\cup B)\cap(A\cup C)$.
				\pause
			\end{itemize}
			\item ``$\supseteq$'': $x\in(A\cup B)\cap(A\cup C)$. Dann liegt x in $(A\cup B)$ und $(A\cup C)$.
			Also liegt x entweder in A oder in (B und C). Folglich gilt 
			$x\in A\cup(A\cap C)$.
			\pause\item Insgesamt ist also $A\cup(B\cap C) = (A\cup B)\cap(A\cup C)$.
		\end{itemize}
	\end{block}
\end{frame}

\begin{frame}{Mengen}
	\begin{block}{Aufgabe aus WS16/17}
		Es sei M eine Menge und es seien $A\subseteq M$ und $B\subseteq M$. Beweisen Sie:
		$M\backslash(A\cap B) = (M\backslash A)\cup(M\backslash B)$
		\begin{itemize}
			\pause \item ``$\subseteq$'': Sei $x\in M\backslash(A\cap B)$. Dann ist $x\in M$ und $x\notin A$ oder $x\notin B$.
			\\Somit ist
			\begin{itemize}
				\item $x\in M$ und $x\notin A$ oder
				\item $x\in M$ und $x\notin B$.
			\end{itemize}
			Also ist $x\in(M\backslash A)$ oder $(M\backslash B)$, folglich also
			$x\in(M\backslash A)\cup(M\backslash B)$.

			\pause\item ``$\supseteq$'': Sei $x\in(M\backslash A)\cup(M\backslash B)$. Dann ist $x\in(M\backslash A)$ oder\\
			$x\in(M\backslash B)$. Somit ist
			\begin{itemize}
				\item $x\in M$ und $x\notin A$ oder
				\item $x\in M$ und $x\notin B$.
			\end{itemize}
			Also ist $x\in M$, und $x\notin A$ oder $x\notin B$. Dann ist $x\in M$ und \\
			$x\notin(A\cap B)$, folglich also $x\in M\backslash(A\cap B)$.
		\end{itemize}
	\end{block}
\end{frame}

\section{Wörter}

\begin{frame}{Wörter}
	\begin{block}{Konkatenation}
		Durch Konkatenation werden einzelne Buchstaben aus einem Alphabet miteinander verbunden.
	\end{block}

	\begin{itemize}
		\pitem Symbol: $\cdot$\pause , also zwei Buchstaben $a$ und $b$ miteinander konkateniert: $a \cdot b$.
		\pitem Nicht kommutativ : $a \cdot b \neq b \cdot a$
		\pitem Aber assoziativ : $(a \cdot b) \cdot c = a \cdot (b \cdot c)$
		\pitem Kurzschreibweise : Ohne Punkte , also $a \cdot b = ab$
	\end{itemize}
\end{frame}

\begin{frame}{Wörter}
	\begin{block}{Wörter: Intuitivere Definition}
		Ein Wort $w$ entsteht durch die Konkatenation durch Buchstaben aus einem Alphabet.
	\end{block}

	$\rightarrow$ Also Abfolge von Zeichen über ein Alphabet A.

	\pause Sei $A := \{a, b, c\}$.

	\begin{itemize}
		\pitem Mögliche Worte: \pause $w_1 := a \cdot b$\pause , $w_2 = b \cdot c \cdot c$\pause , $w_3 = a \cdot c \cdot c \cdot b \cdot a$.
		\pitem Keine möglichen Worte: \pause $d$.
		\end{itemize}
\end{frame}

\begin{frame}{Wörter}
	\begin{block}{Wörter: Abstraktere Definition}
		Ein Wort $w$  über dem Alphabet $A$  ist definiert als surjektive Abbildung  $w : \Z_n \rightarrow A$. Dabei heißt $n$ die Länge $|w|$ des Wortes.
	\end{block}

	\begin{itemize}
		\pitem $\Z_n$  $ = \{i \in \N : 0 \leq i < n \}$
		
		\pause $\Z_3  = \{0, 1, 2\},  \Z_2 = \{0, 1\}, \Z_0 = \emptyset$.
		
		\pitem Länge oder Kardinalität eines Wortes:  $|w|$. \pause $|abcde|$ $= 5$.
		
		\pitem Wort $w = abdec$ als Relation aufgeschrieben: \pause $w = \{(0, a), (1, b), (2, d), (3, e), (4, c)\}$.  Also $w(0) = a, w(1) = b, w(2) = d, \dots$
		
		\pause Damit sieht man auch: $|w| = |\{(0, a), (1, b), (2, d), (3, e), (4, c)\}| = 5$.
	\end{itemize}
\end{frame}

\begin{frame}{Wörter}
	\begin{itemize}
		\item Wort der Kardinalität 0?
	\end{itemize}

	\pause

	\begin{block}{Das leere Wort}
		Das leere Wort $\varepsilon$ ist definiert ein Wort mit Kardinalität 0 , also mit 0 Zeichen.
	\end{block}

	\begin{itemize}
		\pitem Leere Wort wird interpretiert als ``nicht sichtbar'' und kann überall platziert werden\pause : $aabc = a\epsilon abc = \epsilon\epsilon a\epsilon bc \epsilon$.
		\pitem $|\{\epsilon\}|$ \pause $ = 1$ , die Menge ist nicht leer! Das leere Wort ist nicht \emph{nichts}! (Vergleiche leere Menge)
		\pitem $|\epsilon| = 0$.
	\end{itemize}
\end{frame}

\begin{frame}{Mehr über Wörter}
	\begin{block}{$A^n$}
		Zu einem Alphabet $A$ ist $A^n$ definiert als die Menge aller Wörter der Länge $n$ über dem Alphabet $A$.
	\end{block}

	\begin{itemize}
		\pitem Nicht mit Mengenpotenz verwechseln!
		\pitem $A := \{a, b, c\}$\pause , $A^2 = \{aa, ab, ac, ba, bb, bc, ca, cb, cc\}$. \pause $A^1 = A, \pause A^0 = \{\epsilon\}$.
	\end{itemize}
	
	\pause Die Menge aller Wörter  \emph{beliebiger} Länge: \pause
	\begin{itemize}
		\item $A^* := \bigcup_{i \in \N_0} A_i$
		\pitem $A := \{a, b, c\}$ . $aa \in A^* , abcabcabc \in A^* , aaaa \in A^* , \epsilon \in A^*$.
	\end{itemize}
\end{frame}

\begin{frame}{Mehr über Wörter}
	\begin{block}{Wort Potenzen}
		Sich direkt wiederholende Teilworte kann man als Wortpotenz darstellen , daher $w_i^n = w_i \cdot w_i \cdots w_i$ (n $\times$ mal).
	\end{block}

	\begin{itemize}
		\pitem $a^4 = aaaa$\pause , $b^3 = bbb$\pause , $c^0 = \pause \epsilon$\pause , $d^1 = \pause d$.
		\pitem $a^3c^2b^6 \pause = aaaccbbbbbb$.
		\pitem $b \cdot a \cdot (n \cdot a)^2$ \pause $ = banana$.
		\end{itemize}
\end{frame}

\begin{frame}{Übung zu Wörter}
	Sei $A$ ein Alphabet.
	
	\begin{taskblock}{Übung zu Wörter}
		\begin{enumerate}
			\item Finde Abbildung $f: A^* \rightarrow A^*$, sodass für alle $w \in A^*$ gilt: \\\quad $2 \cdot |w| = |f(w)|$.
			\item Finde Abbildung $g: A^* \rightarrow A^*$, sodass für alle $w \in A^*$ gilt: \\\quad $|w| + 1 = |g(w)|$.
			\item Sind $f, g$ injektiv und/oder surjektiv?
		\end{enumerate}
	\end{taskblock}

	\pause
	
	\begin{enumerate}
		\item $f: A^* \rightarrow A^*, w \mapsto w \cdot w$.
		\pitem $g: A^* \rightarrow A^*, w \mapsto w \cdot x, x \in A$.
	\end{enumerate}
\end{frame}

\begin{frame}{Übung zu Wörter}
	\begin{enumerate}
		\item $f: A^* \rightarrow A^*, w \mapsto w \cdot w$.
		\begin{itemize}
			\pitem $f$ ist injektiv\pause , denn jedes $w$ aus der Bildmenge wird von maximal einem Wort abgebildet.
			\pitem $f$ ist nicht surjektiv\pause , denn z.B. bildet nichts auf $x \in A$ ab (oder auf andere Wörter mit ungerader Anzahl an Buchstaben).
		\end{itemize}
		\pitem $g: A^* \rightarrow A^*, w \mapsto w \cdot x, x \in A$.
		\begin{itemize}
			\pitem $g$ ist injektiv.
			\pitem $g$ ist nicht surjektiv\pause , denn z.B. bildet nichts auf $\epsilon$ ab.
		\end{itemize}
	\end{enumerate}
\end{frame}

\section{Formale Sprachen}

\begin{frame}{Formale Sprache}
	\begin{itemize}
		\pitem Was war nochmal $A^*$? Menge aller Wörter \emph{beliebiger} Länge über Alphabet $A$.
	\end{itemize}

	\pause
	
	\begin{block}{Formale Sprache}
		Eine Formale Sprache $L$ über einem Alphabet $A$ ist eine Teilmenge $L \subseteq A^*$.
	\end{block}

	\begin{itemize}
		\pitem Zufälliges Beispiel: \pause $A := \{b, n, a\}$.
		\begin{itemize}
			\pitem $L_1 := \{ban, baan, nba, aa\}$ ist eine mögliche formale Sprache über $A$.
			\pitem $L_2 := \{banana, bananana, banananana, ...\}$ \pause $ = \{w : w = bana(na)^k, k \in \N\}$ auch.
			\pitem $L_3 := \{ban, baan, baaan, ...\}$ auch. \pause Andere Schreibweise? \pause \\ $ L_3 = \{w : w = ba^kn, k \in \N \}$
		\end{itemize}
		\pitem Formale Sprachen sind also nicht zwangsweise endliche Mengen.
		\pitem Praktischeres Beispiel: $A := \{w : w $ ist ein ASCII Symbol $\}$.
		\begin{itemize}
			\pitem $L_4 := \{class, if, else, while, for, ...\}$ ist eine formale Sprache über $A$.
			\pitem $L_5 := \{w : w = a \cdot b$ mit $a$ als Großbuchstabe und $b$ als Groß- oder Kleinbuchstabe $ \} \pause \backslash L_4$ \pause ist eine formale Sprache von korrekten Klassennamen in Java.
		\end{itemize}
	\end{itemize}
\end{frame}

\begin{frame}{Übung zu formalen Sprachen}
	$A := \{a, b\}$
	
	\begin{itemize}
		\pitem Sprache $L$ aller Wörter über $A$, die nicht das Teilwort $ab$ enthalten?
		\begin{itemize}
			\pitem Was passiert wenn ein solches Wort ein $a$ enthält? \pause Dann keine $b$'s mehr!
			\pitem $L = \{w_1 \cdot w_2 : w_1 \in \{b\}^*$ und $w_2 \in \{a\}^* \}$
			\pitem Andere Möglichkeit\pause : Suche Wörter mit $ab$ und nehme diese Weg.
			\pitem $L = \{a, b\}^*\pause \backslash\{ w_1 \cdot ab \cdot w_2 : w_1, w_2 \in \{a,b\}^* \}$
		\end{itemize}
	\end{itemize}
\end{frame}

\begin{frame}{Übung zu formalen Sprachen}
	
	Sei $A := \{a, b\}, B := \{0, 1\}$.
	
	\begin{taskblock}{Aufgabe zu formalen Sprachen}
		\begin{enumerate}
			\item Sprache $L_1 \subseteq A^*$ von Wörtern, die mindestens drei $b$'s enthalten.
			\item Sprache $L_2 \subseteq A^*$ von Wörtern, die gerade Zahl von $a$'s enthält.
			\item Sprache $L_3 \subseteq B^*$ von Wörtern, die, interpretiert als Binärzahl eine gerade Zahl sind.
		\end{enumerate}
	\end{taskblock}

	\pause
	
	\begin{enumerate}
		\item $L_1 = \{w = w_1  b  w_2  b  w_3 b w_4 : w_1,w_2,w_3,w_4 \in A^* \}$
		\pitem $L_2 = \{w = (w_1 a w_2 a w_3)^* : w_1,w_2,w_3 \in \{b\}^* \}$ \pause (Ist da $\epsilon$ drin?)
		\pitem $L_3 = \{w = w \cdot 0 : w \in B^* \}$
	\end{enumerate}
\end{frame}

\input{assets/slide-post}