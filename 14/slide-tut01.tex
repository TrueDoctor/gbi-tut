\def\tutdate{08.02.2018}

\documentclass{beamer}
\usepackage{../templates/beamerthemekit}
%\usepackage{enumitem}
\input{../assets/dependencies}
\input{../assets/definitions}
\title[Grundbegriffe der Informatik]{GBI\\Tutorium 8}
\date{\tutdate}
\subtitle{\tutTitle}
\author{Dennis Kobert dennis@kobert.dev}

\institute{}

\titleimage{bg}
%\titleimage{bg-advent}

%\include{assets/multislider}

\setbeamercovered{invisible}

%\usepackage[citestyle=authoryear,bibstyle=numeric,hyperref,backend=biber]{biblatex}
%\addbibresource{templates/example.bib}
%\bibhang1em

	
\def\tutTitle{Turingmaschinen, Reguläre Sprachen}
\begin{document}

\selectlanguage{ngerman}

%title page
\begin{frame}
	\titlepage
\end{frame}

\section{Turingmaschinen}


\begin{frame}{Definition von Turingmaschinen}
	\begin{block}{Definition von Turingmaschinen}
		Eine Turingmaschine $T = (Z, z_0, X, f, g, m)$ besteht aus:
		\begin{itemize}
			\item $Z$ Zustandsmenge
			\item $z_0 \in Z$ Startzustand
			\item $X$ Bandalphabet
			\item $\Box$ Blanksymbol (sozusagen Markierung für Leerzeichen)
			\item $f: Z \times X \rightarrow Z$ partielle Zustandsübergangsfunktion
			\item $g: Z \times X \rightarrow X$ partielle Ausgabefunktion
			\item $m: Z \times X \rightarrow \{L, N, R\}$ partielle Bewegungsfunktion
		\end{itemize}
	\end{block}

	Anmerkung: partielle Funktionen sind \markGreen{nicht linkstotal}, also manche Elemente des Definitionsbereichs werden nicht abgebildet.
\end{frame}

\begin{frame}{Beispiel einer Turingmaschine}
	\includegraphics[scale=0.5]{images/turingexample_withoutannotations.png}
\end{frame}

\begin{frame}{Halten von Turingmaschinen}
	
	\p
	
	\begin{block}{Halten einer Turingmaschine}
		Wenn eine Turingmaschine in einem Zustand ist, für den das nächste Eingabezeichen durch die Übergangsfunktion $f$ nicht definiert ist, \markGreen{hält} die Maschine.
	\end{block}

	\bp

	Wann hält also eine Turingmaschine \markBlue{nicht}?
	
	\bp
	
	\begin{block}{Nicht-Halten einer Turingmmaschine}
		Wenn eine Turingmaschine in eine endlose Schleife gerät, so \markGreen{hält sie nicht}.
	\end{block}
\end{frame}

\begin{frame}{Entscheidbarkeit}
	\ip
	
	\begin{block}{Durch Turingmaschine akzeptierte Sprache}
		Eine Turingmaschine \markBlue{akzeptiert} eine formale Sprache $L$, wenn sie für jedes Wort $w \in L$ in einem akzeptierenden Zustand hält.
	\end{block}

	\bp

	\begin{block}{Entscheidbare Sprache}
		Eine formale Sprache $L$ heißt \markBlue{entscheidbar}, wenn es eine Turingmaschine gibt, die \markGreen{immer hält} und $L$ akzeptiert.
	\end{block}

	\bp

	\begin{block}{Aufzählbare Sprache}
		Eine formale Sprache $L$ heißt \markBlue{aufzählbar}, wenn es eine Turingmaschine gibt, die $L$ akzeptiert.
	\end{block}
\end{frame}

\begin{frame}{Vom endlichen Akzeptor zur Turingmaschine}
	\begin{taskblock}{Akzeptieren von Turingmaschinen}
		Wie kann man aus dem gegebenen endlichen Akzeptor eine Turingmaschine machen, die dieselbe Sprache akzeptiert?
	\end{taskblock}

	\includegraphics[scale=0.5]{images/AufgAkzeptor2.png}
\end{frame}

\begin{frame}{Lösung}
	Einfach gesagt: mache aus jedem Übergang $a$ einen Turingmaschinen-Übergang der Art $a|a,R$, also bei jedem Zeichen mache den Zustandsübergang, überschreibe aber das Zeichen nicht und gehe zum nächsten Zeichen.
	
	\vspace{.2cm}
	\bp
	
	Formaler ausgedrückt? \pause 
	\begin{itemize}
		\item Für allgemeinen endlichen Akzeptor $(Z, z_0, X, f, Y, h)$, definiere eine Turingmaschine $T := (Z, z_0, X \cup Y, f, g, h)$, also $Z, z_0, f$ gleich und mit Bandalphabet $=$ Eingabealphabet $\cup$ Ausgabealphabet
		\item $g(z, x) := x \quad \forall (z,x)$ in $f$ definiert
		\item $m(z, x) := R \quad \forall (z,y)$ in $f$ definiert		
	\end{itemize}

	\bp
	
	Jeder endliche Akzeptor kann so zu einer Turingmaschine umgeformt werden, die dieselbe Sprache akzeptiert.
\end{frame}

\begin{frame}{Über endliche Akzeptoren hinaus}
	Sei $L$ die Sprache von Palindromen über $\{a,b\}$ ($L = \{aabaa, bbababb, aa, \epsilon\}$). 
	
	\begin{itemize}
		\pitem Ist die Sprache regulär, also gibt es einen endlichen Akzeptor, der diese akzeptiert? \pause Nein.
		\pitem Ist die Sprache entscheidbar, also gibt es eine stets haltende Turingmaschine, die $L$ akzeptiert?
	\end{itemize}
\end{frame}

\begin{frame}{Palindromerkennung mit Turingmaschinen}
	Ja, nämlich: 
		
	\includegraphics[scale=0.4]{images/turingmaschine_palindrome.png}	
\end{frame}

\begin{frame}{Turingmaschinen Entwurfsaufgabe}
	
	\begin{columns}
		\begin{column}{0.65\textwidth}
			\begin{taskblock}{Turingmaschine Entwurf}
				Entwerfe eine Turingmaschine, die...
				\begin{itemize}
					\item als Eingabe eine Binärzahl auf dem Band erhält
					\item als Ausgabe diese Zahl restlos durch zwei teilt und auf dem Band stehen lässt
				\end{itemize}
			\end{taskblock}
		\end{column}
		
		\begin{column}{0.45\textwidth}
			\pause
			
			\includegraphics[scale=0.5]{images/turingmaschine_div.png}
		\end{column}
	\end{columns}

\end{frame}

\begin{frame}{Turingmaschinen Entwurfsaufgabe}
	
	\begin{columns}
		\begin{column}{0.65\textwidth}
			\begin{taskblock}{Turingmaschine Entwurf}
				Entwerfe eine Turingmaschine, die...
				\begin{itemize}
					\item als Eingabe eine Binärzahl auf dem Band erhält
					\item als Ausgabe diese Zahl um eins erhöht auf dem Band stehen lässt
					\item den Kopf der Turingmaschine auf dem ersten Zeichen der Ausgabe stehen hat.
				\end{itemize}
			\end{taskblock}
		\end{column}
		
		\begin{column}{0.45\textwidth}
			\pause
			
			\includegraphics[scale=0.5]{images/turingmaschine_plusone.png}
		\end{column}
	\end{columns}
	
\end{frame}

\begin{frame}{Konfiguration von Turingmaschinen}
	Angenommen, man kennt eine Turingmaschine, hat mit der Abarbeitung eines Wortes angefangen, will aber pausieren, um später weiterzumachen...
	
	\vspace{.2cm}
	
	Was muss man sich alles merken, um später weiter zu machen?
	
	\begin{itemize}
		\pitem Derzeitiger Zustand, in dem die Turingmaschine steht
		\pitem Inhalt des Bandes
	\end{itemize}

	\bp
	\vspace{.2cm}
	
	\begin{block}{Konfiguration von Turingmaschinen}
		Wenn während dem Arbeiten einer Turingmaschine auf dem Band das Wort $w_1 a w_2$ mit $w_1, w_2 \in X^*, a \in X$ steht\ip, der Kopf der Turingmaschine auf das Zeichen $a$ zeigt \ip und die Turingmaschine im Zustand $z$ ist\ip, so schreibt man die \markBlue{Konfiguration der Turingmaschine} als $\Box w_1 (z) a w_2 \Box $.
	\end{block}
\end{frame}

\begin{frame}{Konfiguration von Turingmaschinen}
	Beispiel:
	
	\vspace{.2cm}
	
	\begin{align*}
	\Box\Box\Box\Box\Box\Box\Box\Box abcbabb&\markBlue{d}aabc \Box\Box\Box\Box\Box \\
	&\uparrow \\
	&KOPF
	\end{align*}
	
	\bp
	...sei das Band der Turingmaschine während Abarbeitung der Eingabe, dazu steht sie im Zustand $z_4$.
	
	\vspace{.2cm}
	
	\bp
	Dann sieht sieht die Konfiguration der Turingmaschine so aus: \ip $\Box abcbabb (z_4) daabc \Box$
\end{frame}

\begin{frame}{Dokumentation einer Abarbeitung mit Konfigurationen}
	
	\begin{taskblock}{Aufgabe zu Konfigurationen}
		Gebe alle Konfigurationen der Turingmaschine bei Abarbeitung des Wortes $11$ an.
	\end{taskblock}

	\begin{columns}
		\begin{column}{0.25\textwidth}
			\includegraphics[scale=0.3]{images/turingmaschine_1k.png}
		\end{column}
		
		\begin{column}{0.4\textwidth}
			\begin{description}
				\pause\item[] $\Box (A)11 \Box$
				\pause\item[$\rightarrow$] $\Box X(B)1 \Box$
				\pause\item[$\rightarrow$] $\Box X1(B) \Box$
				\pause\item[$\rightarrow$] $\Box X1 \Box (C) \Box$
				\pause\item[$\rightarrow$] $\Box X1 (D) \Box 1\Box$
				\pause\item[$\rightarrow$] $\Box X (E) 1 \Box 1\Box$
				\pause\item[$\rightarrow$] $\Box (E) X 1 \Box 1\Box$
				\pause\item[$\rightarrow$] $\Box 1 (A) 1 \Box 1\Box$
			\end{description}
		\end{column}
	
		\begin{column}{0.4\textwidth}
			\begin{description}
				\pause\item[$\rightarrow$] $\Box 1 X (B) \Box 1\Box$
				\pause\item[$\rightarrow$] $\Box 1 X \Box (C) 1\Box$
				\pause\item[$\rightarrow$] $\Box 1 X \Box 1 (C)\Box$
				\pause\item[$\rightarrow$] $\Box 1 X \Box (D) 1 1\Box$
				\pause\item[$\rightarrow$] $\Box 1 X (D) \Box 1 1\Box$
				\pause\item[$\rightarrow$] $\Box 1 (E) X \Box 1 1\Box$
				\pause\item[$\rightarrow$] $\Box 1 1 (A) \Box 1 1\Box$
			\end{description}
		\end{column}
	\end{columns}
\end{frame}

\begin{frame}{Halteproblem}
	\markBlue{Halteproblem}: Für einen gegebenen Algorithmus, gelangt dieser bei seiner Abarbeitung zu einem Ende und hält?
	
	\begin{itemize}
		\pitem Algorithmen können durch Turingmaschinen durchgeführt werden
		\pitem Turingmaschinen können durch sogenannte universelle Turingmaschinen simuliert werden
		\begin{itemize}
			\pitem Wenn eine Turingmaschine $T$ kodiert ist mit dem Wort $w$, dann ist $T_w: X \rightarrow X$ eine Funktion, die Eingaben auf die Ausgabe der Turingmaschine $T$ mappt.
			\pitem Also mit $X = \{1,0\}$ gibt z.B. $T_w(100101) = 001$ genau dann zurück, wenn, sofern man $100101$ als Eingabe an die Turingmaschine mit der Kodierung $w$ gibt, diese hält und als Ausgabe $001$ erzeugt.
		\end{itemize}
	\end{itemize}

	\bp
	
	Dann lässt sich das Halteproblem auch als Sprache formulieren:
	
	\vspace{.2cm}
	
	$\quad H = \{w \in A^* : w \text{ ist eine TM-Codierung und } T_w(w) \text{ hält.}\}$
	
	bzw. als allgemeinerer Fall:
	
	$\quad \hat{H} = \{(w,x) \in A^* \times A^* : w \text{ ist eine TM-Codierung und } T_w(x) \text{ hält.}\}$
\end{frame}

\begin{frame}{Halteproblem}
	Das Halteproblem ist unentscheidbar\ip, dh. es gibt keine Turingmaschine, die $H$ entscheidet.
\end{frame}

\begin{frame}{Busy Beaver}
	\markBlue{Busy Beaver TM} ist eine Turingmaschine mit $n$ Zuständen, die möglichst viele Einsen auf das Band schreibt \markBlue{und hält}.
	
	\begin{itemize}
		\pitem Also nicht einfach ewig Einsen aufschreibt und nie aufhört.
	\end{itemize}
	
	\bp
	
	\markBlue{Busy Beaver Problem}: Für eine gegebene Turingmaschine mit $n$ Zuständen, die Einsen aufschreibt und hält: Schreibt sie auch maximal viele Einsen auf?
	
	\bp
	\vspace{.2cm}
	
	Das Busy Beaver Problem ist nicht entscheidbar, bzw. die Busy Beaver Funktion $bb(n)$, die definiert, wieviele einsen von einer Busy Beaver TM maximal geschrieben werden können, ist nicht berechenbar.
	
	\bp
	
	\vspace{.2cm}
	
	Beispielwerte von $bb$: $bb(1) = 1\ip, bb(2) = 4\ip, bb(5) \geq 4098\ip, bb(6) > 10^{18276}$.
\end{frame}

\begin{frame}{Busy Beaver für $n = 3$}
	% 3 zustände und ein haltezustand	
	\includegraphics[scale=0.8]{images/turingmaschine_bb.png}	
\end{frame}

\section{Reguläre Ausdrücke}

\begin{frame}{Regulärer Ausdruck}
\begin{block}{Regulärer Ausdruck}
	\begin{itemize}
		\pitem $Z = \{ |, (,), *, \emptyset\}$ Alphabet von ``Hilfssymbolen''
		\pitem Alphabet $A \setminus Z$
		\pitem Ein \textbf{regulärer Ausdruck} (RA) über $A$ ist eine Zeichenfolge über dem Alphabet $A \cup Z$, die gewissen Vorschriften genügt.
		\pitem Vorschriften
		\begin{itemize}
			\pitem $\emptyset$ ist ein RA
			\pitem Für jedes $x \in A$ ist x ein RA
			\pitem Wenn $R_1$ und $R_2$ RA sind, dann auch $(R_1|R_2)$ und $(R_1R_2)$
			\pitem Wenn R ein RA ist, dann auch $(R*)$
		\end{itemize}
	\end{itemize}
\end{block}
\end{frame}

\begin{frame}{Klammerregeln}
\begin{itemize}
\pitem ``Stern- vor Punktrechnung''
\pitem ``Punkt- vor Strichrechnung''
\pitem[$\rightarrow$]$R_1|R_2R_3*$ Kurzform für $(R_1|(R_2(R_3*)))$
\pitem Bei mehreren gleichen Operatoren ohne Klammern links geklammert
\pitem[$\rightarrow$] $R_1|R_2|R_3$ Kurzform für $((R_1|R_2)|R_3)$
\end{itemize}
\pause
\textbf{Aufgabe}\\
Entferne so viele Klammern wie möglich, ohne die Bedeutung des RA zu verändern.\\
\begin{itemize}
\item $(((((ab)b)*)*)|(\emptyset*))$ \pause $\rightarrow$ $(abb)**|\emptyset*$ \pause
\item $((a(a|b))|b)$ \pause  $\rightarrow$ $a(a|b)|b$
\end{itemize}
\end{frame}

\begin{frame}{Alternative Definition}
Wir können die Syntax von regulären Ausdrücken auch über eine kontextfreie Grammatik definieren. \\ \vspace{0.2cm}
\begin{taskblock}{Aufgabe}Vervollständigt die folgende Grammatik.\end{taskblock}
$G = (\{R\}, \{|,(,),*,\emptyset\} \cup A, R, P)$\\
mit $P =\{R\rightarrow \emptyset, R \rightarrow \pause x \text{ }(\text{mit } x \in A),$\\$
R \rightarrow (R|R), R \rightarrow (RR),$\\$
R \rightarrow(R*)$\\$
R \rightarrow \epsilon\}$

\pause

\vspace{.4cm} 
Wieso brauchen wir $\epsilon$?
\end{frame}

\begin{frame}{Durch R beschriebene Sprache}
\textbf{Notation}\\
\begin{itemize}
\item Spitze Klammern $\langle , \rangle$
\end{itemize}

\pause

\textbf{Regeln}\\
%Hier vielleicht selbst auf die Regeln kommen lassen		
\begin{itemize}
\item $\langle \emptyset \rangle = \pause \{\}$ \pause
\item $\langle x \rangle = \pause \{x\}$ für jedes $x \in A$ \pause
\item $\langle R_1| R_2\rangle = \pause \langle R_1\rangle \cup \langle R_2\rangle$ \pause
\item $\langle R_1 R_1 \rangle = \pause \langle R_1 \rangle \cdot \langle R_2 \rangle$ \pause
\item $\langle R* \rangle = \pause \langle R \rangle *$
\end{itemize}
\end{frame}

\begin{frame}{Charakterisierung regulärer Sprachen}
\begin{block}{Satz}
Für jede formale Sprache $L$ sind äquivalent:
\begin{enumerate}
\item $L$ kann von einem endlichen Akzeptor erkannt werden.
\item $L$ kann durch einen regulären Ausdruck beschrieben werden
\item $L$ kann von einer rechtslinearen Grammatik erzeugt werden.
\end{enumerate}
Solche Sprachen heißten regulär.
\end{block}
\end{frame}

\begin{frame}{Anwendung von regulären Ausdrücken}
\vfill \centering Zum selbst probieren:\\http://regexr.com/\\\vspace{.2cm}Achtung: Reguläre Ausdrücke in praktischer Programmierung funktionieren zwar ähnlich, haben aber eine andere Syntax und können teils mehr!\vfill
\end{frame}

\section{Rechtslineare Grammatiken}
\begin{frame}{Rechtslineare Grammatiken}
\begin{block}{Definition}
Eine rechtslineare Grammatik ist eine reguläre Grammatik $G=(N,T,S,P)$ mit der Einschränkung, dass alle Produktionen die folgende Form haben:
\begin{itemize}
\item $X \rightarrow w$ mit $w \in T^*$ oder
\item $x\rightarrow wY$ mit $w \in T^*$, $Y \in N$
\end{itemize}
\end{block}
\end{frame}

\begin{frame}
\begin{taskblock}{Aufgabe zu rechtslinearen Grammatiken}
Gebe zu $L = \{ w \in \{ 0,1 \}^* | \exists k \in \mathbb{N}_0: Num_2(w) = 2^k + 1 \}$ jeweils einen regulären Ausdruck R und eine rechtslineare Grammatik G an, sodass $L = \langle R \rangle = L(G)$ gilt.
\end{taskblock}
\pause
\textbf{Lösung}\\		
\begin{itemize}
\item $R = (0*10)|(0*1(0)* 1) = \pause 0*10|0*10* 1 $ \pause
\item $G = (\{S,A\}, \{0,1\}, S, \{S \rightarrow0S|10|1A, A \rightarrow 0A|1\})$
\end{itemize}
\end{frame}

\begin{frame}
	\includegraphics[width=\linewidth]{../images/thatsall.png}
\end{frame}


\end{document}