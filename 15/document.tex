\def\tutdate{07.02.2018}

\documentclass{beamer}
\usepackage{../templates/beamerthemekit}
%\usepackage{enumitem}
\input{../assets/dependencies}
\input{../assets/definitions}
\title[Grundbegriffe der Informatik]{GBI\\Tutorium 8}
\date{\tutdate}
\subtitle{\tutTitle}
\author{Dennis Kobert dennis@kobert.dev}

\institute{}

\titleimage{bg}
%\titleimage{bg-advent}

%\include{assets/multislider}

\setbeamercovered{invisible}

%\usepackage[citestyle=authoryear,bibstyle=numeric,hyperref,backend=biber]{biblatex}
%\addbibresource{templates/example.bib}
%\bibhang1em

	
\def\tutTitle{Relationen, Altklausur}
\begin{document}

\selectlanguage{ngerman}

%title page
\begin{frame}
	\titlepage
\end{frame}

\begin{frame}
\section{Relationen}
Schon bekannte Eigenschaften von Relationen: \newline
\begin{itemize}
	\item transitiv ($\forall x, y, z \in M: (x,y) \in R \land (y, z) \in R \rightarrow (x,z) \in R$)
	\item reflexiv ($\forall x \in M: \rightarrow (x,x) \in R$)
	\item symmetrisch ($\forall x, y \in M: (x,y)  \rightarrow (y,x) \in R$)
		\item Äquivalenzrelation ist reflexiv, transitiv und symmetrisch
\end{itemize}
\end{frame}

\begin{frame}

Neue Eigenschaft: \newline
\begin{itemize}
	\item antisymmetrisch ($\forall x, y \in M: (x,y) \in R \land (y, x) \in R \rightarrow x = y$)
	\item Halbordung ist reflexiv, transitiv und antisymmetrisch
\end{itemize}
\end{frame}

\begin{frame}

Welche Eigenschaften sind hier erfüllt? \newline
\begin{itemize}
	\item $\subset$
	\item $<$
	\item $\leq$
	\item Relation R auf $A^*$ mit v R w $\leftrightarrow \exists u: vu=w$
\end{itemize}
\end{frame}

\section{Master-Theorem}
\begin{frame}
	\begin{block}{Auflösung des Mastertheorem}
	\begin{description}
		\item[Fall 1:] Wenn $f \in \okalk(n^{\log_b a -\epsilon})$ für ein
		$\epsilon>0$ ist, dann ist $T\in \Theta(n^{\log_b a})$.
		\item[Fall 2:] Wenn $f \in \Theta(n^{\log_b a})$ ist, dann ist
		$T\in \Theta(n^{\log_b a}\log n)$.
		\item[Fall 3:] Wenn $f \in \Omega(n^{\log_b a +\epsilon})$ für ein
		$\epsilon>0$ ist, und wenn es eine Konstante $d$ gibt mit $0<d<1$, so
		dass für alle hinreichend großen $n$ gilt $af(n/b)\leq d f$, dann
		ist $T\in \Theta(f)$.
	\end{description}
\end{block}
\end{frame}

\begin{frame}{Pseudocode für Mergesort}
\includegraphics[scale=0.5]{Pseudo.png}
\end{frame}

\begin{frame}
	\includegraphics[width=\linewidth]{../images/thatsall.png}
\end{frame}


\end{document}