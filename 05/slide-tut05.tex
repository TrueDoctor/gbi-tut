\def\tutdate{25.11.2019}

\documentclass{beamer}
\usepackage{../templates/beamerthemekitwide}
%\usepackage{enumitem}
\input{../assets/dependencies}
\input{../assets/definitions}
\title[Grundbegriffe der Informatik]{GBI\\Tutorium 8}
\date{\tutdate}
\subtitle{\tutTitle}
\author{Dennis Kobert dennis@kobert.dev}

\institute{}

\titleimage{bg}
%\titleimage{bg-advent}

%\include{assets/multislider}

\setbeamercovered{invisible}

%\usepackage[citestyle=authoryear,bibstyle=numeric,hyperref,backend=biber]{biblatex}
%\addbibresource{templates/example.bib}
%\bibhang1em


\begin{document}

\selectlanguage{ngerman}

%title page
\begin{frame}
	\titlepage
\end{frame}

\section{Übersetzung und Kodierung}

\begin{frame}{Herführung zu Zahlendarstellungen}
	\pause Wir betrachten die Alphabete $A_{dez} := \Z_{10}, A_{bin} := \{0, 1\}, A_{oct} := \Z_8$.
	\begin{itemize}
		\pitem Was können wir daraus machen?
		\pitem $A_{dez}^* \supset \{42, 1337, 999\}$.
		\pitem $A_{bin}^* \supset \{101010, 10100111001, 1111100111\}$.
		\pitem $A_{oct}^* \supset \{52, 2471, 1747\}$.
		\pitem Wir suchen eine Möglichkeit, diese \markGreen{Zahlen} zu \markGreen{deuten}.
		\pitem Aber irgendwie so, dass $42_{\in A_{dez}} \stackrel{Deutung}{=} 101010_{\in A_{bin}} \stackrel{Deutung}{=} 52_{\in A_{oct}}$.
	\end{itemize}
\end{frame}

\subsection{Kodierung von Zahlen}

\begin{frame}{Definition von Zahlendarstellungen}
	\pause
	
	\begin{block}{$Num_k$}
		Einer Zeichenkette $Z_k$ aus Ziffern \p wird mit $Num_k$ eine eindeutige Zahl zugeordnet:
		
		\vspace{.2cm}
		
		\vspace{.2cm}
		
		 \p $Num_k(\epsilon) = 0$
		
		\vspace{.2cm}
		
		 \p $Num_k(wx) = k \cdot Num_k(w) + num_k(x)$ mit $w \in Z_k^*$ und $x \in Z_k$.
	\end{block}

	\pause
	
	\begin{block}{$num_k$}
		Einer einzelnen Ziffer $x \in Z_k$ aus einem Alphabet von Ziffern $Z_k$ wird mit $num_k(x)$ der Wert der Zahl zugewiesen.
	\end{block}

	\pause
	
	\begin{itemize}
		\item Wichtig: $Num_k(w) \neq num_k(w)$!
		\pitem Was ist: $num_{10}(3) \p = 3\p , num_{10}(7) \p = 7\p , num_{10}(11) = \p $ nicht definiert.
		\pitem Für Zahlen $\geq k$: Benutze $Num_k$!
	\end{itemize}
\end{frame}

\begin{frame}{Beispiel zu Zahlendarstellungen}
	$Num_k(\epsilon) = 0$.
	
	$Num_k(wx) = k \cdot Num_k(w) + num_k(x)$ mit $w \in Z_k^*$ und $x \in Z_k$.
	
	\vspace{.3cm}
	
	\p Was ist $Num_{10}(123)$?
	\begin{itemize}
		\pitem $Num_{10}(123) \p = 10 \cdot Num_{10} (12) + num_{10}(3) \p = 10 \cdot ( Num_{10} (1) + num_{10}(2)) + num_{10}(3) \p = 10 \cdot ( num_{10}(1) + 10 \cdot num_{10}(2) ) + num_{10}(3) \p = 10 \cdot ( 1 + 10 \cdot 2 ) + 3 \p = 123$.
	\end{itemize}
	\p Yay?
	
	\p Was ist der dezimale Zahlenwert der Binärzahl 1010? \p Diesmal Basis $k = 2$.
	\begin{itemize}
		\pitem $Num_{2}(1010) \p = 2 \cdot Num_2(101) + num_2(0) \p = 2 \cdot (2 \cdot Num_2(10) + num_2(1) + num_2(0) \p = 2 \cdot (2 \cdot (2 \cdot Num_2(1) + num_2(0) ) + num_2(1) ) + num_2(0) \p = 2 \cdot (2 \cdot (2 \cdot num_2(1) + num_2(0) ) + num_2(1) ) + num_2(0) \p = 2 \cdot ( 2 \cdot (2 \cdot 1 + 0) + 1) + 0) = 10$.
	\end{itemize}
	\p Yay!
\end{frame}

\begin{frame}{Aufgaben zu Zahlendarstellungen}
	$Num_k(\epsilon) = 0$.
	
	$Num_k(wx) = k \cdot Num_k(w) + num_k(x)$ mit $w \in Z_k^*$ und $x \in Z_k$.
	
	\begin{taskblock}{Übungen zu Zahlendarstellungen}
		Berechne den numerischen Wert der folgenden Zahlen anderer Zahlensysteme nach dem vorgestellten Schema:
		\begin{itemize}
			\item $Num_8(345)$.
			\item $Num_2(11001)$.
			\item $Num_2(1000)$.
			\item $Num_4(123)$.
			\item $Num_{16}(4DF)$. (Zusatz)
		\end{itemize}
	\end{taskblock}

	Anmerkung: Hexadezimalzahlen sind zur Basis $16$ und verwenden als Ziffern (in aufsteigender Reihenfolge: $1, 2, 3, 4, 5, 6, 7, 8, 9, A, B, C, D, E, F$.
\end{frame}

\begin{frame}{Aufgaben zu Zahlendarstellungen}
	\pause Lösungen:
	\begin{itemize}
		\pitem $Num_8(345) \p = 229$.
		\pitem $Num_2(11001) \p = 25$.
		\pitem $Num_2(1000) \p = 8$.
		\pitem $Num_4(123) \p = 27$.
		\pitem $Num_{16}(4DF) \p = 1247$. 
	\end{itemize}
\end{frame}

\begin{frame}{Einfachere Umrechnung von Zahlendarstellungen}
	Es gilt: $2(2(2(2(2 \cdot 1 + 0)+1)+0)+1)+0 \p = 2^5 \cdot 1 + 2^4 \cdot 0 + 2^3 \cdot 1 + 2^2 \cdot 0 + 2^1 \cdot 1 + 2^0 \cdot 0$.
	
	\p Daher, einfachere Rechenweise: $Num_k(w) = k^0 \cdot w(0) + k^1 \cdot w(1) + k^2 \cdot w(2) + ...$.
	
	\p Was sind folgende Zahlen in Dezimal im Kopf gerechnet?
	
	\begin{itemize}
		\pitem $Num_2(10101) \p = 21$.
		\pitem $Num_2(11101) \p = 29$.
		\pitem $Num_2(1111111111) \p = 1023$.
	\end{itemize}
	
\end{frame}
		
\begin{frame}{Einfachere Umrechnung von Zahlendarstellungen}
	$Num_k(w) = k^0 \cdot w(0) + k^1 \cdot w(1) + k^2 \cdot w(2) + ...$.
	
	\p Was sind folgende Zahlen in Dezimal im Kopf gerechnet?
	
	\begin{itemize}
		\pitem $Num_{16}(A1) \p = 161$.
		\pitem $Num_{16}(BC) \p = 188$.
		\pitem $Num_{16}(14) \p = 20$.
	\end{itemize}
	
\end{frame}

\begin{frame}{Rechnen mit $div$ und $mod$}
	\pause
	\begin{block}{$div$ Funktion}
		Die Funktion $div$ \markGreen{dividiert ganzzahlig.} \p (Schneidet also den Rest ab).
	\end{block}
	\pause
	\begin{block}{$mod$ Funktion}
		Die Modulo Funktion $mod$ gibt den \markGreen{Rest einer ganzzahligen Division} zurück.
	\end{block}
	\pause
	\begin{itemize}
		\item $22$ div $8 \p = 2 \p $ $(\frac{22}{8} = 2,75)$.
		\pitem $22$ mod $8 \p = 6$.
	\end{itemize}

	\pause Fülle die Tabelle aus:
	
	\begin{tabular}{r | c c c c c c c c c c c c c}
		x & 0 & 1 & 2 & 3 & 4 & 5 & 6 & 7 & 8 & 9 & 10 & 11 & 12\\\hline
		x $div$ 4 & \p 0 & \p 0 & \p 0 & \p 0 & \p 1 & \p 1 & \p 1 & \p 1 & \p 2 & \p 2 & \p 2 & \p 2& \p 3\\
		x $mod$ 4 & \p 0 & \p 1 & \p 2 & \p 3 & \p 0 & \p 1 & \p 2 & \p 3 & \p 0 & \p 1 & \p 2 & \p 3& \p 0\\
	\end{tabular}
\end{frame}

\subsection{Repräsentation von Zahlen}

\newcommand{\definitionOfRepr}{
	\begin{align*}
	\fRepr_k(n) =
	\begin{cases}
	\frepr_k(n) & \text{ falls } n < k \\
	\fRepr_k(n \text{ div } k) \cdot \frepr_k(n \text{ mod } k) & \text{ falls } n \geq k
	\end{cases}
	\end{align*}
}

\begin{frame}{Von Zeichen zu Zahlen zurück zu Zahlen}
	$11101_2$ ist also $29_{10}$. \p Was ist $29_{10}$ in binär? \pause
	\begin{block}{$k$-äre Darstellung}
		Die Repräsentation einer Zahl $n$ \p zur Basis $k$ \p lässt sich wie folgt ermitteln:\p
		\definitionOfRepr
		\p Achtung! \p Das $\cdot$ Symbol steht für Konkatenation, nicht für Multiplikation!
	\end{block}
\end{frame}

\begin{frame}{Beispiel zu $Repr_k$}
	\definitionOfRepr
	
	\pause Zum Beispiel: \p 
	\begin{align*}
	\fRepr_2(29) 
	\only<+->{&= \fRepr_2(29 \fdiv 2) \cdot \frepr_2(29 \fmod 2)}  \\
	\only<+->{&= \fRepr_2(14) \cdot \frepr_2(1) \\}
	\only<+->{&= \fRepr_2(14 \fdiv 2) \cdot \frepr_2(14 \fmod 2) \cdot 1} \\
	\only<+->{&= \fRepr_2(7) \cdot \frepr_2(0) \cdot 1 }\\
	\only<+->{&= \fRepr_2(7 \fdiv 2) \cdot \frepr_2(7 \fmod 2) \cdot 01} \\
	\only<+->{&= \fRepr_2(3) \cdot \frepr(1) \cdot 01 }\\
	\only<+->{&= \fRepr_2(3 \fdiv 2) \cdot \frepr(3 \fmod 2) \cdot 101} \\
	\only<+->{&= \fRepr_2(1) \cdot \frepr(1) \cdot 101} \\
	\only<+->{&= 11101}
	\end{align*}
\end{frame}
\newcommand{\uhd}{_{16}}

\begin{frame}{Beispiel zu $Repr_k$}
	\definitionOfRepr

	\pause Beispiel mit Hexadezimalzahlen: \p 
	\begin{align*}
		\fRepr\uhd (29) 
		\only<+->{&= \fRepr\uhd (29 \fdiv 16) \cdot \frepr\uhd (29 \fmod 16)}  \\
		\only<+->{&= \fRepr\uhd (1) \cdot \frepr\uhd (13)}\\
		\only<+->{&= 1 \cdot D = 1D}
	\end{align*}
\end{frame}

\begin{frame}{Übung zu $Repr_k$}
	\definitionOfRepr
	\begin{taskblock}{Übung zu $Repr_k$}
		Berechne die Repräsentationen folgender Zahlen in gegebenen Zahlensystemen:
		\begin{itemize}
			\item $\fRepr_2(13)$.
			\item $\fRepr_4(15)$.
			\item $\fRepr\uhd (268)$.
		\end{itemize}
	\end{taskblock}

	\pause Lösungen:
	\begin{itemize}
		\pitem $\fRepr_2(13) \p = 1101$.
		\pitem $\fRepr_4(15) \p = 33$.
		\pitem $\fRepr\uhd (268) \p = 10C$.
	\end{itemize}
\end{frame}

\subsection{Zweierkomplement-Darstellung}

\begin{frame}{Feste Länge von Binärzahlen}
	\pause
	
	\begin{block}{$bin_\ell$}
		Die Funktion $\fbin_{\ell}\colon \Z_{2^{\ell}} \to \{0,1\}^{\ell}$ \p  bringt eine gegebene Binärzahl auf eine feste Länge\p , indem sie mit Nullen vorne aufgefüllt wird. \p Formell wird sie definiert als:\p
		\begin{align*}\fbin_{\ell}(n) = 0^{\ell- |\fRepr_2(n)|} \fRepr_2(n)\end{align*}
	\end{block}

	\pause Beispiel:
	\begin{itemize}
		\pitem $\fbin_8(3) \p = 0^{8 - |\fRepr_2(3)|}\fRepr_2(3) \p = 0^{8 - |11|}\cdot 11 \p = 0^{8 - 2} \cdot 11 \p = 0^6 \cdot 11 = 00000011$.
		\pitem $\fbin_{16}(3) \p = 0000000000000011$.
	\end{itemize}
\end{frame}

\newcommand{\definitionOfZkpl}{
\begin{align*}
\fZkpl_{\ell}(x) =
\begin{cases}
0 \fbin_{\ell-1}(x) & \text{ falls } x\geq 0 \\
1 \fbin_{\ell-1}(2^{\ell-1} + x) & \text{ falls } x< 0 
\end{cases}
\end{align*}
}

\begin{frame}{Zweierkomplement}
	\p Was ist mit negative Zahlen?
	
	\begin{itemize}
		\pitem Idee: Verwende das erste Bit, um zu speichern, ob die Zahl positiv oder negativ ist.
		\pitem Beispiel: \p $5 = \markGreen{0}101_{zkpl}$\p , $-5 = \markGreen{1}011_{zkpl}$.
	\end{itemize}

	\pause

	\begin{block}{Zweierkomplement Darstellung}
	Die Zweierkomplementdarstellung einer Zahl $x$ \p mit der Länge $\ell$ ist wie folgt definiert:\p
	\definitionOfZkpl		
	\end{block}

	\begin{itemize}
		\pitem Wieso $\ell - 1$?
	\end{itemize}
	
\end{frame}

\begin{frame}{Aufgaben zu Zweierkomplement-Darstellung}
	\definitionOfZkpl
	
	Was sind folgende Zahlen in Zweierkomplement-Darstellung?
	\begin{itemize}
		\item $\fZkpl_4 (3)$ \visible<1>{$=0011$.}
		\item $\fZkpl_4 (7)$ \visible<2>{$=0111$.}
		\item $\fZkpl_4 (-5)$ \visible<3>{$=1011$.}
		\item $\fZkpl_8 (13)$ \visible<4>{$=0000 1101$.}
		\item $\fZkpl_8 (-34)$ \visible<5>{$=1101 1110$.}
		\item $\fZkpl_8 (-9)$ \visible<6>{$=1111 0111$.}
	\end{itemize}
\end{frame}

\section{Übersetzungen}

\begin{frame} {Übersetzungen}
	\begin{block} {Definition der Semantikabbildung}
		Sei \textit{Sem} die Menge der Bedeutungen. \p Ferner seien $A$ und $B$ Alphabete \p und $L_A \subseteq A^* \text{ und } L_B \subseteq B^*$.\\
		\p Weiter sei $sem_A:L_A \rightarrow Sem$ \p und $sem_B: L_B \rightarrow Sem$\\
		\p Dann heißt $f: L_A \rightarrow L_B$ Übersetzung \p , wenn gilt: für jedes $w \in L_A$ gilt $sem_A(w) = sem_B(f(w))$.
	\end{block}
	\begin{itemize}
		\pitem Bedeutungserhaltende Abbildungen von Wörtern auf Wörter
	\end{itemize}
	\textbf{Beispiel}\\
	\p Betrachte $Trans_{2,16}: \mathbb{Z}_{16}^* \rightarrow \mathbb{Z}_{2}^*$ mit $ Trans_{2,16}(w) = Repr_2(Num_{16}(w))$
	\begin{itemize}
		\pitem $Trans_{2,16}(A3) = Repr_2(Num_{16}(A3)) = Repr_2(163) = 10100011$
	\end{itemize}
\end{frame}
\begin{frame}{Wozu Übersetzungen}
	\begin{itemize}
		\pitem Lesbarkeit (vergleiche $DF_{16}$ mit $11011111_2$)
		\pitem Verschlüsselung
		\pitem Kompression (Informationen platzsparend aufschreiben)
		\pitem Kontextabhängige Semantiken (Deutsch $\rightarrow$ Englisch)
		\pitem Fehlererkennung
	\end{itemize} 
\end{frame}


\begin{frame}{Codierungen}	
	\begin{block}{Definitionen}
		\begin{itemize}
			\pitem Codewort $f(w)$ \p einer Codierung $f: L_A \rightarrow L_B$
			\pitem Code: $\{f(w)|w \in L_A\} = f(L_A)$
			\pitem Codierung: \textbf{Injektive} Übersetzung
			\begin{itemize}
				\pitem Ich komme immer eindeutig von einem Codewort f(w) zu $w$ zurück
			\end{itemize}
		\end{itemize}
	\end{block}\p
	\textbf{Bemerkung}\\
	\begin{itemize}
		\pitem Was ist, wenn $L_A$ unendlich ist (man kann nicht alle Möglichkeiten aufzählen)
		\pitem Auswege: Homomorphismen, Block-Codierungen
	\end{itemize}
\end{frame}


\subsection{Homomorphismen}

\begin{frame}{Homomorphismen}
	\begin{block} {Definition von Homomorphismen}\p
		Seien $A, B$ Alphabete. \p Dann ist $h: A^* \rightarrow B^*$ \p ein Homomorphismus\p , falls für alle $w_1, w_2 \in A^*$ gilt:\\ \p
		\begin{equation*}
		h(w_1w_2) = h(w_1)h(w_2)
		\end{equation*}
	\end{block}

	\begin{itemize}
		\pitem Ein Homomorphismus ist Abbildung, die mit Konkatenation verträglich ist
		\pitem Homomorphismus ist $\varepsilon$-frei, wenn für jedes $x \in A: h(x) \neq \varepsilon$
		\pitem Homomorphismen lassen das leere Wort unverändert, also $h(\varepsilon) = \varepsilon$
	\end{itemize}
\end{frame}

\begin{frame}
	Sei $h$ ein Homomorphismus.
	
	\begin{taskblock}{Übung zu Homomorphismen}
			\begin{enumerate}
				\pitem $h(a) = 001$ und $h(b) = 1101$. Was ist dann $h(bba)$? 
				\pitem[$\rightarrow$] $h(bba) = h(b)h(b)h(a) = 1101 \cdot 1101 \cdot 001 = 11011101001$
				\pitem Sei $h(a) = 01, h(b) = 11 \text{ und } h(c) = \varepsilon$. Nun sei $h(w)= 011101$. Was war $w$? 
				\pitem[$\rightarrow$] $aba$ oder $cabccac$, ... Allgemein: $w \in \{c\}^* \cdot \{a\} \cdot \{c\}^* \cdot \{b\} \cdot \{c\}^* \cdot \{a\} \cdot \{c\}^*$ \\ \p $\epsilon$-Freiheit hat also die Eindeutigkeit zerstört!
				\pitem Kann h aus 2 eine Codierung sein?
				\pitem[$\rightarrow$] Nein, da nicht injektiv!
				\pitem Warum will man $\varepsilon$-freie Homomorphismen?
				\pitem[$\rightarrow$] Information geht sonst verloren!
				\pitem Was heißt hier ``Information geht verloren''? 
				\pitem[$\rightarrow$] Es gibt $w_1 \neq w_2$ mit $h(w_1) = h(w_2)$
			\end{enumerate}
	\end{taskblock}
\end{frame}

\begin{frame}
	\begin{itemize}
		\pitem Information kann auch anders ``verloren'' gehen
		\pitem[$\rightarrow$] z.B. $h(a) = 0, h(b) = 1, h(c) = 10$ \p -- Wie das?
	\end{itemize} \pause
	\begin{block}{Präfixfreiheit}
		\p Gegeben ist ein Homomorphismus $h: A^* \rightarrow B^*$.\\
		\p Wenn für keine zwei verschiedenen $x_1, x_2 \in A$ gilt\p , dass $h(x_1)$  Präfix von $h(x_2)$ ist\p , dann ist $h$ präfixfrei. 
	\end{block}
	\pause
	\begin{block}{Satz}
		Präfixfreie Codes sind injektiv.
	\end{block} \pause
	\textbf{Beispiele}\\
	\begin{itemize}
		\pitem $h(a) = 01 \text{ und } h(b) = 1101 $ ist präfixfrei
		\pitem $g(a) = 01 \text{ und } g(b) = 011$ ist nicht präfixfrei
	\end{itemize}
\end{frame}

\subsection{Huffman Codierung}

\begin{frame}{Huffman-Codierung}
	\begin{itemize}
		\pitem Komprimiert eine Zeichenkette
		\pitem Kodiert häufiger vorkommende Zeichen zu kürzeren Codewörter als Zeichen die seltener vorkommen.
		\pitem Vorgehensweise:
		\begin{enumerate}
			\pitem Zähle Häufigkeiten aller Zeichen der Zeichenkette
			\pitem Schreibe alle vorkommenden Zeichen und ihre Häufigkeiten nebeneinander
			\pitem Wiederhole, bis der Baum fertig ist:
			\begin{itemize}
				\pitem Verbinde die zwei Zeichen mit niedrigsten Häufigkeiten zu neuem Knoten über diesen
				\pitem Dieser hat als Zahl die aufsummierte Häufigkeiten
			\end{itemize}
			\pitem Danach: Alle linken Kanten werden mit $0$ kodiert, alle rechten Kanten mit $1$
		\end{enumerate}
	\end{itemize}

	\p Das Ergebnis ist eine Zeichenkette aus $\{0,1\}$\p , die kürzer ist als die ursprüngliche Zeichenkette in binär.
\end{frame}

\begin{frame}{Huffman-Codierung}
	Gegeben
	\begin{itemize}
		\item $w \in A^*$ 
		\only<1>{ \\ }\textbf{w } = \texttt{ afebfecaffdeddccefbeff }
		\pause
		\item Anzahl der Vorkommen aller Zeichen in w ($N_x(w)$)
	\end{itemize}		
	\only<2>{
		\textbf{Häufigkeiten:}\\
		\begin{tabular}{c c c c c c c}
			\hline
			x &a&b&c&d&e&f\\
			\hline
			$N_x(w)$& 2& 2&3&3& 5& 7\\
			\hline
		\end{tabular}							
	}
	\pause
	Zwei Phasen zur Bestimmung eines Huffman-Codes
	\begin{enumerate}
		\item Konstruieren eines ``Baumes''
		\begin{itemize}
			\item Blätter entsprechen den Zeichen
			\item Kanten mit 0 und 1 beschriften\\ 
			\only<3>{
				\includegraphics[scale=0.8]{../images/Baum.PNG}
				\begin{tabular}{c c c c c c c}
					\textbf{Häufigkeiten:}\\
					\hline
					x  &a&b&c&d&e&f\\
					\hline
					$N_x(w)$& 2& 2&3&3& 5& 7\\
					\hline
				\end{tabular}	
			}
		\end{itemize} 
		\pause
		\item Ablesen der Codes aus dem Baum (Pfadbeschriftungen)
	\end{enumerate}
	\only<4>{
		\includegraphics[scale=0.46]{../images/Baum.png}
		\hspace{0.4cm}
		\begin{tabular}{c c c c c c c}
			\textbf{Häufigkeiten:}\\
			\hline
			x &a&b&c&d&e&f\\
			\hline
			$N_x(w)$ & 2& 2&3&3& 5& 7\\
			
			\textbf{Codewörter:}\\
			\hline
			x &a&b&c&d&e&f\\
			\hline
			h(x)& 000& 001&100&101& 01& 11\\
			\hline
		\end{tabular}							
	}		
\end{frame}

\begin{frame}{Übung zu Huffman Codierung}
	\begin{taskblock}{Übung}
		Sei $A = \{$\texttt a, b, c, d, e, f, g, h$\}$\\
		\begin{itemize}
			\item Codiere das Wort \texttt{badcfehg} mit Hilfe der Huffman-Codierung \pause
			\item [$\rightarrow$]Mögliche Lösung: 001 100 010 011 101 000 111 110
			\pause
			\item Wie lauten die Codewörter, wenn für das Wort $w$ gilt: $N_a(w) = 1, N_b(w) = 2, N_c(w) = 2, N_d(w) =8, N_e(w) =16, N_f(w) =32, N_g(w) = 64, N_h(w) = 128$
			
		\end{itemize} \pause
		Mögliche Lösung:\\
		\begin{tabular}{|c|c c c c c c c c|}
			\hline
			x &a&b&c&d&e&f&g&h\\
			\hline
			h(x)& 0000000& 0000001&000001&00001& 0001&001& 01&1\\
			\hline
		\end{tabular}
	\end{taskblock}
\end{frame}	

	\begin{frame}
	\begin{itemize}
		\item Wie lang wäre das zweite Wort (\texttt{abbcccc} $\texttt{d}^{8}$...$\texttt{g}^{64}\texttt{h}^{128}$) mit dem ersten Code codiert? 
		\pause
		\item[$\rightarrow$] 741 Symbole. Also dreimal so lang wie das Original. \pause
		\item Wie lang wäre das zweite Wort mit dem zweiten Code codiert?\pause
		\item[$\rightarrow$] 501 Symbole. Also nur zweimal so lang wie das Original. \pause
		\item Was fällt euch auf?
	\end{itemize}
\end{frame}		

\begin{frame}{Wahr oder falsch?}
	Sei $h: A^* \rightarrow \mathbb{Z}_2$ eine Huffman-Codierung
	\begin{itemize}
		\item h ist ein $\varepsilon$-freier Homomorphismus \pause \textbf{Wahr!}\pause
		\item Häufigere Symbole werden mit langen Worten codiert, seltene mit kürzeren \pause \textbf{Falsch!}\pause
		\item Die Kompression ist am stärksten, wenn die Häufigkeiten aller Zeichen ungefähr gleich sind. \pause \textbf{Falsch!} \pause
		\item h ist präfixfrei \pause \textbf{Wahr!} \pause
		\item Es kann noch kürzere Codierungen geben \pause \textbf{Falsch!}
	\end{itemize}
\end{frame}

	
\begin{frame}{Huffman-Codierung}
	\begin{block}{Eigenschaften}
		Sei $A$ ein Alphabet und $w \in A$. Dann gilt für die Huffman-Codierung h:
		\begin{itemize}
			\item $h: A^* \rightarrow \mathbb{Z}_2$
			\item $h$ ist $\varepsilon$-freier Homomorphismus
			\item $h$ ist präfixfreier Homomorphismus
			\item Häufigere Symbole werden mit kurzen Worten codiert, seltene mit längeren
			\item Produziert kürzestmögliche Codierungen
		\end{itemize}
	\end{block}
\end{frame}

\begin{frame}{Block-Codierung mit Huffman}
	\begin{itemize}
		\pitem Wir betrachten nicht mehr einzelne Symbole, sondern Blöcke von fester Länge $b > 1$
		\pitem Blätter des Huffman-Baums sind jetzt \textit{Wörter der Länge b}
	\end{itemize}

	\vspace{.5cm}
	
	Beispiel an der Tafel: Codierung von $aab\cdot deg \cdot deg \cdot aab \cdot ole \cdot aab \cdot deg \cdot aab$.\p
	
	\vspace{.5cm}
	
	\p
	\begin{itemize}
		\pitem Alphabet $A =\{$\texttt{a,b,c,d} $\}$
		\pitem Text über $A$, der nur aus Teilwörtern der Länge 10 zusammengesetzt ist, in denen jeweils immer nur ein Symbol vorkommt
		%\item z.B. \texttt{aaaaaaaaaabbbbbbbbbbcccccccccc}...
		\pitem Angenommen $\texttt{a}^{10}$, ..., $\texttt{d}^{10}$ kommen alle gleich häufig vor. Wie lang ist dann die Huffman-Codierung? \pause
		\pitem[$\rightarrow$] Ein Fünftel, weil jeder Zehnerblock durch zwei Bits codiert wird
	\end{itemize}
	
\end{frame}


\begin{frame}
	\includegraphics[width=\linewidth]{../images/thatsall.png}
\end{frame}

\end{document}
